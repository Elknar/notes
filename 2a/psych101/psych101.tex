\documentclass[12pt]{article}
\usepackage{amsmath,amssymb,parskip,custom}
\usepackage[margin=.6in]{geometry}

\begin{document}

\title{PSYCH 101 - Introduction to Psychology}
\author{Kevin James}
\date{\vspace{-2ex}Fall 2013}
\maketitle\HRule

\section*{Outline}
Psychology is the analytical study of the interactions within society and of the thoughts and behaviours of every person, giving us a less naive understanding of those aspects of our culture. An understanding of psychology helps one to gain a better understanding of oneself and of others.

\section*{ABCs of Psychology}
The ABCs are the three most important aspects of psychology to which all topics can be brought back to or described in terms of. They include:
\begin{itemize}
\item {\bf Affect} -- {\it feelings, moods, and states}, including the study of emotional interactions.
\item {\bf Behaviour} -- {\it actions and performances}. This is slightly more overt than the study of emotions, since actions are often, if not always, overt, whereas emotions can be hidden.
\item {\bf Cognition} -- {\it thoughts, decisions, attitudes}. This deals with decision-making and analysis.
\end{itemize}

\section*{Basic Model of Psychology}
We begin by looking at a person and his features, then his behaviours. We then look at the outcomes "generated" by this person, as determined by his features and behaviours, and attempt to draw inferrences between them. We must also take into account the person's environment -- not only physically, but socially, emotionally, and mentally.

In effect, the basic model is {\bf person} $\rightarrow$ {\bf behaviour} $\rightarrow$ {\bf outcome}.

\section*{Perspectives of Psychology}
\subsection*{Psychodynamic}
The {\bf psychodynamic approach} was heralded by Sigmund Freud. This perspective deals with forces at play within the subconscious psyche of a person: these forces are said to be constantly in flux and determine one's thoughts and behaviours.

\subsection*{Behaviourism}
Mostly led by John Watson, {\bf behaviourism} was originally referred to as the {\bf environmental perspective}. John Watson was very much in favor of the hard sciences, as opposed to Freud's idea of psychology, and thus focused on overt behaviours -- ie. things which can be measured. Thus, behaviourism deals with a person's overt actions and {\it behaviours}. It also heavily features the "reward model", ie. that people do things because they will be rewarded in some faction.

\subsection*{Cognitive}
The {\bf cognitive model} of psychology, whose most famous leader was Antonia Bandura, was a more centrist view of psychology. The cognitive perspective deals primarily with thoughts and perceptions, and thus the behaviours they cause. This is very different from the psychodynamic model mainly because it deals with the {\it conscious} mind: we can actually ask "why did you come to that decision?" and find ourselves with a somewhat satisfactory answer. Students of the cognitive model tended to study animals such as lab rats, and apply their behaviour upon humans.

\subsection*{Biological}
The {\bf biological perspective} has been around since the time of Freud and is somewhat related to the psychodynamic approach. This approach was created by biologists studying the brain independantly of psychological studies. This perspective can be characterized by the question: "what are my neurons doing when I do or think \_\_\_\_\_\_\_\_?".

\subsection*{Neuropsychology, Evolution, Genetics}
{\bf Evolutionary psychology} also comes from the study of biologiy, and deals with the question of how human thoughts and behaviours have evolved and changed over history. This tends to be highly speculative, as... well, thoughts do not fossilize.

\subsection*{Socio-Cultural}
The {\bf socio-cultural approach} is a more modern approach which focuses on how the environment plays a role in a person's thoughts and behaviour. This deals heavily with the interactions between and within groups, societies, and cultures, and is expanded on by the field of sociology.

This approach also deals with the different perspectives of different cultures -- for example, Eastern psychology is completely different from the topcs this course will cover.

\section*{Science of Psychology}
Psychology, as a discipline, is seen as somewhat inferior when compared to the hard sciences, in that it tends not to be regarded as having as much rigour. Regardless, psychology follows the same process as any hard science: ie. observing by watching. After all, Einstein said that "the whole of Science is nothing more than the refinement of everyday thinking".

\subsection*{Goals: Understand, Explain, Predict, and Control}
The goals of science, as layed out by Carl Poper, are to {\it understand, explain, predict, and control} a certain area of interest. Studying some topic gives us a greater understanding of said topic, which allows us to explain that topic, which can hopefully allow us to predict future events, and most ideally to control those events.

\subsection*{Refinement: from Theory to Theory}
Refinement of our theories tends to follow the following cycle: develop theory, form hypothesis, carry out observation, analyze results, refine theory. Repeat {\it ad nauseum}.

\subsection*{Criteria: Empirical, Replicatible, and Falsifiable}
We have certain criteria which we follow to ensure the science of psychology has any use:
\begin{itemize}
\item Our observations must be empirical, ie. quantifiable and measurable. If all observations are subjective, then we are forming opinions, or at best we are performing philosophy.
\item Our experiments must be replicable by other people under other circumstances. If we declare something to be true, but that result can never be repeated, then it has no use whatsoever. Again, we have simply formed a useless -- and, anyway, inaccurate -- opinion instead of following the scientific method.
\item Falsifiability is important, though this is not readily apparant. For any theory to be accepted as a scientific theory, it must be possible to prove that theory wrong. This doesn't mean that this theory must necessary {\it be} wrong, simply that there must be some potential observation which could convince us of the invalidity of this theory.
\end{itemize}
"Truth is arrived at by the painstaking process of eliminating the untrue"

\subsection*{Associations: Correlation and Causation}
We can have {\bf direct correlations}, which imply that one variable increases as does an other, or {\bf negative orrelations}, which imply the opposite relationship. Correlation, though, does not imply causation, and as such does not tell us which effect is {\it causing} the other.

If we find that A and B are directly correlated, this could mean $A \implies B$, $B \implies A$, or $X \implies A \land B$.

\section*{Origins of Psychology}
It is generally believed that {\bf Willhelm Wundt} was responsible for creating the field of Psychology in 1879 when an experiment of his concluded that being aware of one's own awareness takes longer than simply allowing the awareness complete control.

\subsection*{Schools of Thought}
\subsubsection*{Structuralism}
Structuralism was founded by Wundt's student {\bf Edward Bradford Titchener}. It focuses on breaking down one's experiences into the most basic components by having subjects engage in introspection.

\subsubsection*{Functionalism}
Functionalism was formed in response to Structuralism by focusing on the work of Charles Darwin and {\bf William James}. This school of thought tries to explain mental processes systematically and accurately by focusing on the purpose of both consciousness and behaviours rather than the elements of said consciousness. Functionalism also takes into account individual differences to a much higher degree than other schools.

\subsubsection*{Behaviourism}
Behaviourism was formed between 1920 and 1960 by American scientists (e.g. {\bf Watson} and {\bf Skinner}) who dismissed introspection and reintroduced psychology as the {\it scientific study of observable behaviour}. They believed in observation, and thus cared much more about poeples actions that the thoughts or feelings behind them.

\subsubsection*{Humanistic Psychology}
Humanistic Psychology rebelled against Freudian psychology and behaviourism by emphasizing the importance of environmental influence with regards to our personal growth. It also emphasized the importance of having our needs for love and acceptance satisfied.

\subsubsection*{Cognitive Neuroscience}
By the 1960s, psychology as a field moved back toward the study of the function of the mind and various mental processes. At this point, cognitive neuroscience was developed. Cognitive Neuroscience: the interdisciplinary study of brain activity linked with cognition.

\subsection*{Psychology Today}
Today, we define psychology as the intesection of behaviour and mental processes.

The various perspectives are as follows:
\begin{itemize}
\item {\bf Neuroscience}: how the body and brain enable emotions, memories, and sensory experiences
\item {\bf Evolutionism}: how the natural selection of traits prompoted the survival of genes
\item {\bf Behavioural Genetics}: how our genes and environment influence individual differences
\item {\bf Psychodynamic}: how behavious springs from unconscious drives and conflicts
\item {\bf Behavioural}: how we learn observable responses
\item {\bf Cognitive}: how we encode, process, and store information
\item {\bf Socio-cultural}: how behaviour and thinking vary cross-culture
\end{itemize}

\section*{Nature vs Nurture}
The issue of nature vs nurture - the relative contributions of biology and experience -  is an oft-discussed one. {\bf Plato} assumed that character and intelligence are largely inherited and that certain ideas are inborn. {\bf Aristotle} countered that there is nothing in the mind that does not first come in from the external world through the senses. {\bf John Locke} rejected the notion of inborn ideas, suggesting that the mind is a blank sheet on which experience writes. {\bf Rene Descartes} disagreed, believing that some ideas are innate. 20 years later, Descartes gained support from a naturalist: {\bf Charles Darwin}.

\section*{Three levels of Analysis}
The three levels of analysis form from an integrated bio-psycho-social approach. Each provides a different but incomplete view.

\begin{itemize}
\item Biological
\begin{enumerate}
\item Natural selection of adaptive traits
\item Genetic predispositions related to the environment
\item Brain mechanisms
\item Hormonal influences
\end{enumerate}
\item Psychological Influences
\begin{enumerate}
\item Learned fears and expectations
\item Emotional responses
\item Cognitive processing and perceptual interpretations
\end{enumerate}
\item Socio-Cultural Influences
\begin{enumerate}
\item Presence of others
\item Cultural, social, and familial expectations
\item Peer and group influences
\item Compelling models, e.g. the media
\end{enumerate}
\end{itemize}

\section*{Affective Development}
\subsection*{Psychodynamic}
The psychodynamic analysis of affective development heavily emphasizes sex and attachment, i.e. the primary biological need is sex and the secondary emotional need is attachment. Freud was a supporter of this theory: he believed that attachment followed from sexual fulfillment of needs.

\subsection*{Behaviourism}
Behaviourists believed the primary biological need was survival, followed by the emotional need of attachment. A child would learn to associate certain people with good things, and develop affectation and attachment as a side-effect of the initial survival mechanism.

\subsubsection*{Harry Harlow's Monkeys}
{\bf Harry Harlow} was a behaviourist who worked with monkeys. He initially took baby monkeys from their mother, and noted that they formed emotional bonds with whatever they had (e.g. blankets, bottles, etc). From this, he wondered if attachment was not a secondary need but a primary one.

One of his studies is among the most famous in history: two surrogate mothers provide for baby animals. He divides the children into two groups: one with a cloth mother and the other with a wire mother which gives milk. The infants are then given both mothers and asked to chose. The babies preferred the cloth mothers, regardless of circumstance. In effect, they chose the mother with which they could snuggle and cling to over the one providing sustenance. Harlow's conclusion: that we have a primary need to form bonds with members of our species, and that love is a necessity.

\subsubsection*{Mary Ainsworth's Response}
{\bf Mary Ainsworth} wanted to know whether Harlow's results would prove similar with humans. She thus created the {\bf strange situation}.

There exist two chairs in a room containing toys, one for a child's mother and the other for a complete stranger. The children would maintain a firm grip to their mother and only begin exploring the room cautiously after a period of time.

When the stranger enters the room, the child would return to it's mother. When the mother leaves, the child would cry and / or go to the mother's chair.

These results varied according to {\bf attachment patterns}:
\begin{itemize}
\item Secure - the child is secure in it's relationship with it's mother
\item Ambivalent - the child is ambivalent about it's situation with it's mother
\item Avoidant - the child is actively hostile in it's relationship with it's mother
\end{itemize}

Note that these relationships go both ways, i.e. both of the people involved in the relationship can be in different situations.

\begin{table}[ht]
\centering
\begin{tabular}{r|ccc}
  & Secure & Ambivalent & Avoidant \\ \hline
  Secure & smooth, reciprocal & secure, tolerant, caring & aggresive, intolerant \\
  Ambivalent & x & hot and cold & domainant / submissive \\
  Avoidant & x & x & power struggle, mistrust
\end{tabular}
\end{table}

\subsubsection*{Albert Bandura}
{\bf Albert Bandura} was a behaviouralit who noted that one way to learn is by observing associations. He called this {\bf trial and error learning}. His realizations: that children acquire a massive amount of information extremely quickly, much of it through social experinces. We learn from other's experinces, through {\bf imitative} or {\bf vicarious learning}.

\section*{Developmental Psychology}
Developmental psychology focuses on three main issues:
\begin{enumerate}
\item {\bf Nature vs Nurture}: how does genetic inheritance and experience influence our development?
\item {\bf Continuity vs Stages}: is development a gradual process or is it a sequence of discrete stages?
\item {\bf Stability vs Change}: do early personality traits persist through life ot do we become a different person through aging?
\end{enumerate}

Developmental psychologists study the physical, mental, and social changes throughout the life span, beginning at conception.

{\bf Habituation} is a decrease in response with repeated stimulation. The more often a stimulation is presented, the less frequently it is used. The seeming boredom with stimuli can be used to evaluate what infants see, hear, feel, etc. This is referred to as the {\bf novelty-preference procedure}, which introduces new things along with many old things.

\subsection*{Brain Development}
The developing brain overproduces neurons an dneural networks which enable walking, talking, etc. Fiber pathways supporting language and agility proliferate into puberty until a {\it pruning process} shuts down excess connections and strengthens others.

\subsection*{Maturation}
Deprivation or abuse can retard development, experiences of takling and reading can speed it up. Maturation sets the basic course of development to be adjusted by further experiences.

\subsubsection*{Motor Development}
The developing brain slowly enables physical coordination; in no small parts based on one's genetics (ex: identical twins first sit up on nearly the same day). Maturation, including the rapid development of the cerebellum, allows us to walk around age one.

\subsubsection*{Infant Memory}
{\bf Infantile Amnesia} is the condition where children beneath the age of three are unable to recall otherwise memorable occasions. The earliest conscious memory forms at around age three-and-a-half.

The nervous system, though, is capable of remembering what the brain cannot: children who had crossed the age-three-border provided physical responses to people they had met before that age.

\subsection*{Cognitive Development}
{\bf Cognition} refers to all the mental activities associated with thinking, knowing, remembering, and communicating.

Developmental psychologist {\bf Jean Piaget} was intrigued by children's inability to answer certain questions, and eventually concluded that children reasoned differently than adults due in part to their brains being completely diffrent than their adult counterparts.

He concluded that a child's mind develops in stages: the maturing brain builds schemas (i.e. concepts or mental molds) into which we pour our experiences. By adulthood, we have built countless schemas for each and every possible topic (e.g. animals, love, spicy food).

Piaget proposed two new concepts: first, we {\bf assimilate} new experiences, interpreting them in terms of our current experiences and preformed schemas. As we adjust with the world, though, we {\bf accomodate} out schemas to accept the information new experiences provide. Broad views are quickly rpelaced by narrowing categoried and developing new schemas as necessary.

Children, then, construct their understanding of the world slowly, and with both periods of change and cognitively stable plateaus:

\begin{itemize}
\item {\bf Age 0-2} {\it "sensorimotor"}: experience the world through senses and actions. First experiences with object permanence and stranger anxiety. Infants live in the present ({\it "out of sight, out of mind"}) and don't understand that objects exist when not perceived.
\item {\bf Age 2-7} {\it "peroperational"}: represent things as words and images, uses intuition rather than logical reasoning. First experiences with pretend play and egocentrism. Children lack principles such as {\bf conservation}, the idea that quantity remains the same despite changes in shape, and have difficulty perceiving things from another's point of view.
\item {\bf 7-11} {\it "concrete operational"}: thinking logically about concrete events, grasping conrete analogies, preforming arithmetical operations. First experiences with conservations, mathematics, transformations. Children here begin to grasp conservation and first gain the mental ability to comprehend math.
\item {\bf 12-adult} {\it "formal operational"}: abstract reasoning. First experiences with abstract logic, potential for mature moral reasoning. {\it If this then that} reasoning is now possible (i.e. consequentialism). Around this age, the concept of a {\it self} is discovered.
\end{itemize}

\subsection*{Theory of the Mind}
Children begin to infer other's thoughts when they form a {\bf theory of the mind} (coined by {\bf David Premack} and {\bf Guy Woodruff}). They can then tease, empathize, and persude others. They also begin to realise that people may hold false beliefs. Note that those with autism to not develop this.

\section*{Social Development}
At around the age of eight months, children develop {\bf stranger anxiety}. They greet strangers by crying and reaching for familiar figures. This is because they have developed a schema for faces.

\subsection*{Origins of Attachment}
The attachment bond is a survival impulse which keeps children near to their caregivers. Though developmental psychologists initially suggested that infants became close to those who satisfied their physical needs, this was eventually proven false.

From Harlow's experiments, we conclude that children become attached to those who were close to them and comfortable. Rather than nourishment, comfort and familiarity is given precedence.

\subsubsection*{Familiarity}
Contact is only one key to attachment; another is familiarity. For many animals, these bonds must form during a certain critical period, i.e. age when these bonds can be formed. {\bf Konrad Lorenz} explored the {\bf imprinting} process. Even baby birds could imprint upon other species, and once formed an imprint is always difficult to reverse. Children, however, do not imprint, they simply become attached to what they've known.

\subsection*{Attachment Differences}
When placed in the strange situation, roughly 60\% of infants display {\it secure attachment}: when the mother is present they are content, but they are distressed otherwise and welcome her return. Other infants show {\it insecure attachment}: they cling to their mother and remain severely upset when she leaves.

Children have inborn differences, so their behaviour is not dependant solely on parenting. While our capacity for love grows as we age and our pleasure at touching and holding never ceases, early attachment does gradually relax.

{\bf Eirk Erikson} concluded that securely attached children approach life with a sense of basic trust, the feeling that the world is predictable and reliable. He attributed this to early parenting: infants with loving caregivers form a lifelong attitude of trust rather than of fear.

Many researchers believe that our early attachments form the foundation for our adult relationships and for our comfort levels with affection and intimacy. Adult styles of romantic love tend to exhibit secure, trusting attachment; insecure, anxious attachment; or avoidance of attachment. These styles affect relationships with children, as avoidant people find parenting stressful and unsatisfying.

\subsubsection*{Deprivation of Attachment}
If secure attachments lead to social competence, the deprivation of love and nurturing can lead to withdrawn, frightened, and even speechless children.

Extreme trauma also seems to have an effect on the mind. CHanges in the brain chemical seratonin (which calms aggressive impulses) can be noted among such cases. Stress can set off a ripple of hormonal changes that permanently wires a childs brain to cope with a malevolent world.

Abuse victims are at considerable risk for depression if they carry a certain gene variation which spurs stress-hormone reduction. Once again, this is a case of a particular environment interacting with particular genes.

\subsubsection*{Disruption of Attachment}
In studies of children separated from their caregivers, initial difficulty tends to quickly give way to detachment. This is a process which follows from seperation distress through sadness and emotional detachment to normality. Separation distress is often difficult for adults as well.

Day care generally has no negative effects on children; regardless of the care, children need a consistent and warm relationship with people whom they can learn to trust.

\subsection*{Parenting Styles}
There exist three major parenting styles:
\begin{itemize}
\item {\bf Authoritarian} parents impose rules and expect obedience.
\item {\bf Permissive} parents submit to their children's desires: they make few demands and use litte punishment.
\item {\bf Autoritative} parents are both demanding and responsive. They exert control by setting and enforcing rules, but they also explain the reasons behind those rules. They encourage open discussion when making rules and permitting exceptions.
\end{itemize}

Children with the highest self-esteem, self-reliance, and social competence come from autoritative parents. Authoritarian parents lead to insecurity and low self-esteem and permissive parents lead to aggression and immaturity.

\section*{Adolescence}
Developing occurs during the entirety of one's life, not only during childhood. During adolescence, this starts with the beginnings of physical maturity and ends with the social acheivement of independant adult status.

\subsection*{Cognitive Development}
\subsubsection*{Developing Reasoning Power}
Reasoning is initially self-focused. Eventually, we reach the intellectual peak and become more capable of abstract reasoning. Adolescents reach the point of debating human nature, absolute truth, and justice.

\subsubsection*{Developing Morality}
A crucial task during development is acheiving a sense of right and wrong. Much of our morality is gut-level, from which we rationalize out beliefs. {\bf Lawrence Kohlberg} sought to describe the development of moral reasoning, the thinking that occurs as we consider right and wrong.

He posed moral dilemmas to adults and children and recorded the results, thus determining that the following three morality schemas exist:
\begin{itemize}
\item {\bf Preconventional morality}: before the age of nine, most children's morality is centered around themselves; they obey rules to avoid punishment or to gain rewards.
\item {\bf Conventional morality}: by early adolescence, children begin to focus on caring for others and upholding laws simply because those laws exist.
\item {\bf Postconventional morality}: The third level of moral reasoning is one that some people never reach: actions are judged to be {\it right} because they follow either from people's rights or from self-defined, basic ethical principles.
\end{itemize}

\subsection*{Social Development}
Each stage of development has its own psychosocial task, a crises which needs resolution. For young children, this is {\it trust, autonomy,} and {\it initiative}. Older children struggle with {\it competence}.

\subsubsection*{Erikson's Stages of Psychosocial Development}
\begin{enumerate}
\item Infancy (0 to 1): Trust vs mistrust. Needs are dependably met, a basic sense of trust is developed.
\item Toddler (1 to 3): Autonomy vs shame and doubt. Learn to exercise their will, do things for themselves or doubt their abilities.
\item Preschool (3 to 6): Initiative bs guilt. Learn to begin tasks or feel guilty about efforts to be independant.
\item Elementary (6 to 12): Industry vs inferiority. Learn the pleasure of applying themselves to a task or feel inferior.
\item Adolescenxe (12 to 17): Identity vs role confusion. Refine the sense of self, test roles and integrate them to form a single identity or become confused about who they are.
\item Young adult (17 to 25): Intimacy vs isolation. Struggle to form close relationships and gain the capacity for intimate love or feel socially isolated.
\item Middle adult (25 to 45): Generativity vs stagnation. Discover a sense of contributing to the world, usually through family and work, or feel that they lack purpose.
\item Late adult (45 to 70): Integrity vs despair. When reflecting on life, an older adult may feel a sense of success or failure.
\end{enumerate}

\subsubsection*{Forming an Identity}
TO refine their sense of identity, adolescents distinguish themselves to create an identity. Social identity often forms from distinctiveness (only woman in a group, only ethnicity in a group, etc).

\subsection*{Parent and Peer Relationships}
Today, delayed independance and earlier sexual maturity has widened the once brief interlude between biological maturity and social independence. FOr those still in school, the time from 18 to the mid-twenties is being increasingly referred to as {\bf emerging adulthood}. Westerners typically ease their way into adulthood, returning to their parents while in school and depending on them financially.

\section*{Neuropsychology}
\subsection*{Interacting with the Environment}
There are two requirements we have for interacting with our environment: \textbf{personal} and \textbf{environmental}. A process that involves bringing something inward is an {\it afferent} process. To detect the external world is not enough for survival, we need to be able to respond back to the environment effectively depending on the demands of that world. Processes that are outward bound are called {\it efferent} processes.

\subsection*{The Neuron}
A neuron is formed of the following substructures:
\begin{itemize}
\item {\bf Dendrites} receive messages from other cells
\item {\bf Axon} passes message away from the cell body to other neurons, muscles, or glands.
\item {\bf Terminal Branches of Axons} form junctions with other cells.
\item {\bf Myelin Sheath} covers the axon of some neurons and helps to speed neural impulses.
\item {\bf Cell Body} is the cell's life support center.
\end{itemize}

The \textbf{dendrite fibers} receive information and conduct it toward the cell body. From there the cell's axon passes the message to other neurons or to muscles or glands. \textbf{Axons} can be up to feet long within the body. The \textbf{myelin sheath} insulates the axons of some neurons and help speed their impulses. Degeneration of the myelin sheath leads to multiple schlorosis.

Neurons transmit messages when stimulated by signals from our senses or when triggered by chemical signals from neighbouring neurons. At such times, a neuron fires an impulse called the \textbf{action potential}: a brief electrical charge that travels down the axon.

Neurons generate electricity from chemical events; this chemistry-to-electricity process involves the exchange of ions. The fluid interior of a resting axon has an excess of negatively charged ions while the fluid outside the axon membrane has positively charged ions. This positive-outside/negative-inside state is called the \textbf{resting potential}.

An axon's surface is selectively permeable. A resting axon has gates that block positive sodium ions. When opened, the membrane is flooded, which depolarizes that section of the axon and causes the next section to open.

Each neuron is a decision making device performing complex calculations as it receives signals from other neurons. Most of these signals are excitatory, others are inhibitory. If excitatory minus inhibitory signals exceed a minimum intensity threshold, the combined signal trigger an action potential.

There also exist multiple types of neurons, each of which with different functions:
\begin{itemize}
\item Sensory neurons carry messages in from the body's sensory receptors to the CNS for processing. Afferent direction from outside CNS to inside CNS. Quantity: 2-3 million.
\item Motor neurons carry instructions OUT from the CNS to the body's muscles and glands. Efferent direction from inside CNS to outside. Qyantity: 2-3 million.
\item Interneruons in brain/spinal cord process information between the sensory input and motor output. Within CNS Neuron to neuron. Quantity: 10-100 {\it billion}.
\end{itemize}

\subsection*{How Neurons Communicate}
The meeting point between neurons is called a \textbf{synapse}. Axon terminal of one neuron is separated from the receiving neuron by a \textit{synaptic gap}. When an action potential reaches the terminals at an axon's end, it triggers the release of chemical messengers, called \textbf{neurotransmitters}.

The neurotransmitter unlocks tiny channels at the receiving site, and electrically-charged atoms flows in, exciting or inhibiting the receiving neuron's readiness to fire. Then in a process called \textbf{reuptake}, the sending neuron reabsorbs the excess neurotransmitters.

\subsection*{Neuronal Connection}
{\bf Luigi Galvani} was famous for work with electrocuting animals to discover the neural networks within their minds.

{\bf Otto Loewi} believed that it was a chemical which sent messages in the brain, whereas electrical impulses would arc and thus not be capable of such precision. It takes him 17 years before he comes up with his experiment (which came to him in a dream): two hearts in two beakers. He takes the liquid from the first and pours it over the second. The liquid alone caused the second heart to beat, therefore there must have been some chemical reaction: this was the fist neurotransmitter.

\subsection*{The Hodgkin-Huxley Model}
{\bf Hodgkin} and {\bf Huxley} presented modern "electro-chemical" theory in 1952, and received Nobel Prize in Physiology for their work. They presented the idea that there existed a chemical process between neurons and an electrical one within neurons. Neurotransmitters would accumulate on the dendrites as they pass. The dendrites check for the correct molecular structure to excite or inhibit.

\subsection*{Agonist and Antagonist Molecule}
An \textbf{agonist} molecule fills the receptor site and actives it, acting like the neurotransmitter. An \textbf{antagonist} molecule fills the lock so that the neurotreansmitter cannot get in and activate the receptor site.

The neurotransmitters can follow multiple potential paths after synaptic transmission:
\begin{itemize}
\item decomposed by certain enxymes
\item taken back up into the sending neuron to be used again ({\bf reuptake})
\item released then binded again and again ({\bf continuous binding})
\end{itemize}

\section*{Neurotransmitters}
\subsection*{How Neurotransmitters Influence Us}
\textbf{Acetylcholine} is one of the best understood neurotransmitters. In addition to its role in learning and memory, it is the messenger at every junction between a motor neuron and skeletal muscle.

{\bf Candace Pert} and {\bf Solomon Snyder} discovered that the brain creates natural occurring opiates. \textbf{Endorphins} help explain good feelings such as the runner's high, the painkilling effect of acupuncture, and the indifference to pain in some severely injured people.

\subsection*{Acetylcholine}
Involved in muscle action, learning, and memory. Deterioration of ACH neurons is implicated in Alzheimer's disease.

\subsection*{Dopamine}
Influence movement, learning, attention, and emotion. An oversupply is linked to schizophrenia; an undersupply to Parkinson's Disease.

\subsection*{Seratonin}
Infuences mood, hunger, sleep, and arousal. Undersupply linked to depression.

\subsection*{Epinephrine Norepinephrine}
Influence alertness, arousal, mood. Same chemical used in endocrine system (adrenaline and noradrenaline).

\subsection*{GABA}
A major inhibitory neurotransmitter. An undersupply is linked to seizures, tremors, and insomnia.

\subsection*{Glutamate}
A major excitary neurotransmitter involved in memory. An oversupply can overstimulate that brain, producing migraines and seizures.

\subsection*{Endorphins}
Known as nature's painkiller, natural morphine. Also involved in various emotions in the limbic system.

\subsection*{Foreign Chemicals}
When flooded with artificial opiates, such as heroine and morphine, the brain stops producing its own. The brain then becomes deprived and uncomfortable.

Drugs and other chemicals affect brain chemistry at synapses, often by amplifying or blocking a neurotransmitter's activity. An \textit{agonist} molecule can mimic the effects of a neurotransmitter.

\textit{Antagonists} block a neurotransmitter's functioning.

\subsubsection*{Cocaine and amphetamines}
Stimulants that increase the release of norepinephrine and block reuptake of dopamine. Perceived as pleasurable and associated behaviours are reinforced.

\subsubsection*{Opiates (opium, morphine, heroine, codeine)}
Agonists that mimic endorphins by attaching to their binding sites.

\subsubsection*{Alcohol}
Generally depresses neural acticity throughout brain.

\section*{The Neurology of Addiction}
The brain maintains homeostasis. In other words, it increases or decreases the production of neurotransmitters to compensate for the effect of addictive substances. We thus need an increased dosage to experience the same effect, once the brain learns how to compensate. The brain learns to anticipate consumption and makes pre-emptive adjustments: this leads to the potential for overdose.

\section*{The Nervous System}
Our \textbf{nervous system} is the body's electrochemical communication network. The brain and spinal cord form the \textbf{central nervous system}, which communicates with the body's sense receptors, muscles and glands via the \textbf{peripheral nervous system (PNS)}.

The PNS information travels through axons that are bundled into electrical cables known as \textbf{nerves}. Information travels in the nervous system through sensory neurons, motor neurons, and interneurons.

\subsection*{Peripheral Nervous System (PNS)}
Our peripheral nervous system has two main components: somatic and autonomic. Our \textbf{somatic nervous system} enables voluntary control of our skeletal muscles.

\subsection*{Nervous System}
The nervous system has multiple aspects:
\begin{enumerate}
\item Peripheral
  \begin{enumerate}
  \item Autonomic (controls self-regulated actions of internal organs and glands)
    \begin{enumerate}
    \item Sympathetic (arousing)
    \item Parasympathetic (calming)
    \end{enumerate}
  \item Somatic (controls voluntary movements of skeletal muscles)
  \end{enumerate}
\item Central (brain and spinal cord)
\end{enumerate}

Our \textbf{autonomic nervous system} controls our glands and the muscles of our internal organs, influencing functions such as glandular activity, heartbeat, and digestion.

The \textbf{sympathetic nervous system} arouses and expends energy. If something alarms, enrages, or challenges you, the sympathetic system will accelerate your heartbeat, raise your blood pressure, slow your digestion, etc.

When stress subsides, your \textbf{parasympathetic nervous system} produces opposite effects. It conserves energy as it calms you by decreasing heartbeat, lowering blood sugar, etc.

\subsubsection*{The Central Nervous System}
The \textit{spinal cord} is an information highway connecting peripheral nervous systems to the brain. Ascending neural fibers send up sensory information and descending fibers send back motor-control info.

The neuropathways governing our \textbf{reflexes}, our automatic responses to stimuli, illustrate the spinal cord's work. The hand jerks away from hot surface before we feel pain because the pain reflex pathway runs through spinal cord to our nervouss ystem but takes longer to get to brain.

\section*{Endocrine System}
Interconnected with the nervous system is endocrine system, the glands of which secrete \textbf{hormones}.

Some hormones are chemically identical to neurotransmitters. While nervous system mesages move quickly, the endocrine system take several seconds or more. The messages linger, however, such as when a person is upset.

In moments of danger, \textbf{adrenal glands} release epinephrine, norepinephrine (adrenaline), and noradrenaline. These hormones increase heart rate, blood pressure, etc, providing us with a surge of energy.

The most influential endocrine gland is the \textbf{pituitary gland}. Controlled by the hypothalamus, this gland releases hormones that influence growth and the release of hormones. It is the master gland which triggers sex glands to release sex hormones.

\section*{The Brain's Electrical Activity}
An electroencephalogram (EEG) is an amplified read-out of the electrical activity in the brain's billions of neuron sweeps. It identifies the electrical wave evoked by a stimulus.

\subsection*{Neuroimaging Techniques}
The \textbf{PET (Positron Emission Tomography scan)} depicts brain activity by showing each brain area's consumption of sugar glucose. After consuming radioactive glucose, a PET tracks the glucose throughout the brain. This shows which areas of brain are most active as a person performs mathematical calculations, looks at images of faces, or daydreams.

An \textbf{MRI (Magnetic Resonance Imaging system)} puts the head in a strong magnetic field, aligning the spinning atoms of brain molecules. A radio wave disorients the atoms, and when they return to their normal spin, signals are released which outline the brain's soft tissue.

\subsection*{The Brainstem}
The \textbf{brainstem} is the brain's oldest and innermost region. It begins where the spinal cord swells, called the \textbf{medulla}, and controls heartbeat and breathing. Just above the medulla is the \textit{pons}, which helps to coordinate movement.

The brainstem is a crossover point where nerves from each side of the brain connect with the body's opposite side. Inside the brainstem, between ears, is the \textbf{reticular formation}, a finger-shaped network of neurons that extends from the spinal cord right up to the thalamus.

The reticular formation is involved in arousal, incites a coma if severed, and produces alertness when stimulated.

\subsection*{Thalamus}
The {\bf thalamus} sits at top of brainstem and acts as brain's sensory switchboard. Receives information from all senses except smell and routes it to the higher brain regions. Directs higher brain replies, sending them to medulla and cerebellum.

\subsection*{Cerebellum}
The {\bf cerebellum (little brain)} extends from the rear of the brainstem and enables one type of nonverbal learning and memory. Helps us judge time, modulate our emotions, discriminate sounds and textures.

These functions all occur without conscious effort, our brain processes most information outside of our body's awareness.

\section*{The Limbic System}
At the border between the brain's older parts and the cerebral hemispheres is the limbic system. One limbic system is the \textit{hippocampus}, which processes memory. The {\bf amygdala} influences aggression and fear.

\subsection{The Hypothalamus}
The {\bf hypothalamus (below the thalamus)} is an important link in the chain of command governing bodily maintenance. Some clusters influence hunger, others thirst, body temperature, sexual behaviour...

The hypothalamus monitors blood chemistry and takes orders from other parts of the brain. For example, thinking about sex could cause hormones to be released, illustrating how the brain influences the endocrine system, which in turn influences the brain.

Stimulating the hypothalamus of animals can trigger their reward system and thus release dopamine or specific centers associated with eating, drinking, sex.

\subsection*{Cerebral Cortex}
A thin surfance of interconnected neural cells, the {\bf cerebral cortex} is the brain's thinking crown, the ultimate control and information processing center.

The larger the cortex, the greater the capacity for learning and thinking, enabling them to be more adaptable. What makes us distinctively human mostly arises from complex functions of our cerebral cortex.

\subsubsection*{Structure of the Cortex}
Supporting the billions of nerve cells in the cortex are nine times as many spidery \textbf{glial cells}, or glue cells. Glials work for neurons, providing nutrients and insulating myelin, mopping up ions and neurotransmitters.

Einstein had more glial cells than the average person.

Each hemisphere of the cortex is divided into four lobes, geographic regions separated by prominent fissures. There are the \textbf{frontal lobes}, behind the forehead, \textbf{parietal lobes}, at the top and to the rear, and the \textbf{occipital lobes}, at the back. Just above ears, find \textbf{temporal lobes}.

The arch-shaped region at the back of the frontal lobe, running from ear-to-ear across the top of the brain is the \textbf{motor cortex}. Body areas requiring precise control require greatest amount of cortical space.

\subsubsection*{Sensory Functions}
The area at the front of the parietal lobes, parallel and just behind motor cortex, is called the \textbf{sensory cortex}. The more sensitive the body region, the larger the sensory cortex area devoted to it.

We also have auditory cortex in temporal lobes, and visual cortex in occipital lobes.

\subsection*{Association Areas}
Three quarters of the thin, wrinkled layer are a part of the association areas whose neurons integrate information. They link sensory inputs with stored memories.

We don't only use a tenth of our brains, rather these association areas interpret and act on info processed by the other areas.

Association areas are found in all 4 lobes, enabling judgement, planning, and processing of new memories in the frontal lobes. With ruptured frontal lobes, we become less inhibited and more judgments seem unrestrained by normal emotions.

The parietal lobes enable mathematical and spacial reasoning. An area on the underside of the right temporal lobe enables us to recognize faces.

\section*{The Brain's Plasticity}
Brains are sculpted by our experiences. Neurons do not usually regenerate when severed, rather they can reorganize in response to damage.

\textit{Constraint-induced therapy} forces the brain to rewire to use a bad limb when the good limb is made unavailable.

Damaged brain functions can migrate to other brain regions, as in the case of those who have lost one of their senses. If a blind person uses one finger to read Braille, the brain area dedicated to that finger expands as the sense of touch invades the visual cortex that normally helps people see.

Adjacent regions in the brain, such as the arm and hand invade one another if one is lost. Touching an arm when a person has lost a hand may cause them to feel as though their hand has been touched as well. The arm area had invaded the space vacated by the hand.

\textbf{Neurogenesis} is the process of generating new neurons. Natural promoters of neurogenesis include exercise, sleep, and non-stressful but stimulating environmnents.

\section*{Our Divided Brain}
In 1960, it was believed that the left hemisphere was the dominant or major hemisphere and the right was subordinate or minor.

In 1961, neurosurgeons speculated that epileptic seizures were caused by an amplification of abnormal brain activity bouncing back and forth between the two cerebral hemispheres. They tried to end this by severing the \textbf{corpus callosum}, the wide band of axon fibers connecting the hemispheres and carrying messages between them.

After severed, people were surprisingly normal, operating fine with these \textbf{split brains}. With a split brain, both hemispheres can comprehend and follow an instruction simultaneously.

These studies revealed that the left hemisphere is more active when a person deliberates over decisions. The right hemisphere understands simple requests, easily perceives objects, and is more engaged when quick, intuitive responses are needed.

The right side is skilled at perceiving emotion and portraying emotions through the more expressive left side.

\subsection*{Right/Left Differences in the Intact Brain}
Perceptual tasks increase glucose consumption in the right side. Speaking or calculating increases activity in the left side.

If the left side is disabled, we lose control of right arm and speech. Sign language, like hearing using the left side to process speech, also uses left hemisphere for sign language. Language is the same to the brain, whether spoken or signed.

While the left side is adept at making quick, literal interpretations of language, the right excels in making inferences.

Gien the word {\it foot}, the left brain will quickly associate it with {\it heel}. But given {\it foot, cry, glass}, the right would more quickly associate it with {\it cut}.

The right hemisphere also orchestrates the sense of self. After right brain damage, we have difficulty perceiving who other people are in relation to oneself. People had difficulty recognizing themselves in photos with the right brain disabled.

\subsection*{Brain Organization and Handedness}
Almost all right-handed people process speech primarily in the left hemisphere, which tends to be larger.

\section*{Behaviour Genetics and Evolutionary Psychology}
\textbf{Behaviour geneticists} study our differences and weigh the effects and interplay of heredity and environment.

Identical twins who develop from a single fertilized egg that splits in two are genetically identical. Fraternal twins develop from separate fertilized eggs; they share a fetal environment but they are no more genetically similar than any two siblings.

\subsection*{Biological vs Adoptive Relatives}
Adoption creates two groups:
\begin{itemize}
\item {\bf Genetic relatives} are biological parents and siblings
\item {\bf Environmental relatives} are adoptive parents and siblings
\end{itemize}

The finding from studies of adoptive families show that people who grow up together, whether biologically related or not, do not resemble one another much in personality.  In traits such as extraversion and agreeableness, adoptees are more similar to their biological parents than to their caregiving adoptive parents.

The environment shared by a family's children has virtually \textbf{no} discernible impact on their personalities. Two adopted children are no more likely to share personalities with each other than with the children down the street.

While genetics may limit the family's environment's influence on personality, parents do influence their children's attitudes, values, manners, faith, and politics.

\section*{Temperment and Heredity}
Infants' temperments are their emotional excitability, whether reactive, intense, and fidgety or easygoing, quiet, and placid. Temperment differences tend to persist.

Heredity predisposes temperment differences: identical twins have more similar personalities than fraternal twins.

One form of a gene that regulates the neurotransmitter serotonin predisposes a fearful temperment in children. We conclude that biologically rooted temperment helps form our enduring personality.

\subsection*{Heritability}
Using twin and adoption studies, behaviour geneticists can mathematically estimate the \textbf{heritability} of a trait, the extent to which variation among individuals can be attributed to differing genes.

If the heritability intelligence is half, thie does \textbf{not} mean that your intelligence is 50 percent genetic. Rather it means that genetic influence explains 50 percent of observed behaviour among people. We can \textbf{never} say what percentage of an individual's personality or intelligence is inherited. Heritability refers instead to the extent to which \textit{differences among people} are attributable to genes.

As environments become more similar, heredity as a source of differences becomes more important and apparent.

If all people were in the same environment, heritability would increase. At the other extreme, if people had similar heredities but were raised in widely different environments, heritability would be much lower.

\subsubsection*{Group Differences}
If genetic influences helps explain individual diversity, the same cannot be said of group differences between men and women or between races. Individual height and weight are highly heritable, yet nutrition influenced these factors more than a century worth of genes.

Heritable individual differences do not imply heritable group differences. If some individuals are genetically disposed to be more aggressive than others, that does not explain why some groups are more aggressive than others.

\subsubsection*{Nature vs Nurture}
Among our similarities, the most important is our enormous adaptive capacity. Some human traits, such as having two eyes, develop in virtually every environment. However, go barefoot for a summer and you develop callused feet, a biological adaption.

Our shared biology enables our developed diversity through adaptation.

\section*{Gene Environment Interaction}
To say that genes and experience are both important is true. But more precisely, they \textit{interact}: environments trigger gene activity. Biological appearances have social consequences and our future environments are the result of our inherent personality, thus  we have not nature versus nurture but rather nature via nurture.

\section*{The New Frontier: Molecular Genetics}
Behaviour geneticists draw on "bottom-up" \textbf{molecular genetics} as it seeks to identify \textit{specific genes} influencing behaviour. The goal of molecular behaviour genetics is to find some of the many genes that influence nromal human traits, such as body weight, sexual orientation, extraversion, etc.

Genetic tests can now reveal at-risk populations for many diseases.

\section*{Evolutionary Psychology: Understanding Human Nature}
Geneticists explore the genetic and environmental roots of human differences. \textbf{Evolutionary psychologists} instead focus on what makes us so much alike to humans. They use Darwin's principle of \textbf{natural selection} to understand the root of behavior and mental processes.

Natural selection works as follows:
\begin{enumerate}
\item Organisms' varied offspring compete for survival.
\item Certain biological and behavioural variations increase their reproductive and survival chances in their environment.
\item Offspring that survive are more likely to pass their genes to ensuing generations.
\item Thus, over time, population characteristics may change.
\end{enumerate}

\section*{Natural Selection and Adaptation}
Nature has selected advantageous variations from among the \textbf{mutations} (random error in gene replication) and from the new gene combinations produced at conception. Genes and experience together wire the brain. Our adaptive flexibility in responding to different environments contributes to our \textit{fitness}, our ability to survive and reproduce.

\subsection*{Evolutionary Success Helps Explain Similarities}
Our shared human traits across cultures were shaped by natural selection acting over the course of human nature.

Our behaviour and biological similarities arise from our shared human genome. No more than 5 percent of the genetic differences among humans arise from population group differences.

\subsection*{Outdated Tendencies}
We are predisposed to behave in ways that promoted our ancestors' surviving and reproducing. We love the taste of sweets and fats, which were hard to come by.

\subsection*{An Evolutionary Explanation of Human Sexuality}
We like sex. This is obvious, but the reason for this is based on the cultural necessity of passing along one's genes.

\section*{ESP}
Extrasensory perception is a psuedo-scientific field which deals with people having more senses or perceptions that the average individual. Nearly half of Americans believe we are capable of this. There exist multiple types of ESP:
\begin{itemize}
\item {\bf Precognition} is the accurate prediction of future events
\item {\bf Clairvoyance} is the direct mental perception of a state of physical affairs
\item {\bf Telepathy} is the direct communication between one mind and another through the use of psi
\item {\bf Psychokinesis} is the ability to use your mind to cause external effects, such as levitating a table
\end{itemize}

\subsection*{The Evidence}
Most evidence for this sensation tends to be anecdotal or based on baised reporting. Despite the will we have to believe this is true, we have not been able to sientifically support, replication, or identify this despite 150 years of research.

\section*{Dreams}
\subsection*{Sleep Onset}
At the onset of sleep, many changes overtake us:
\begin{itemize}
\item Physiological
  \begin{itemize}
  \item HR slows
  \item breathing more irregular
  \item muscles relax (sometimes with a sudden twitch or jerk)
  \item sensory equipnment closes down (vision first, then hearing, and then others)
  \item hynogogic jerk and myolonic kick, may have experienced it if you've slept with someone
  \end{itemize}
\item Neurological
  \begin{itemize}
  \item Electrical voltage increases with more diffuse firings throughout brain
  \end{itemize}
\item Psychological
  \begin{itemize}
  \item awareness of environment and time slips away
  \item control of thought and imagery decreases
  \item hypnagogic hallucinations
  \end{itemize}
\item Brief, weird, and unusual experiences just before nodding off
\item Tend to be surprising, involve movement, and have red as the dominant colour
\item Most common hallucination involves falling or stepping out into space
\end{itemize}

Dreams occur during periods with \textbf{REM} sleep. During REM sleep, heart rate rises and breathing becomes rapid. Sleep paralysis occurs when the brainstem blocks the motor cortex's messages and the muscles don't move. This is sometimes known as paradoxical sleep. The brain is active but the body is immobile.

Genitals are aroused, although not as a result of sexual stimulation.

The length of REM sleep increases the longer you remain asleep. With age, there are more awakenings and less deep sleep.

\subsection*{Physiology of Dreams}
REM occurs in infants and all mammals. RAS become active prior to and during REM, which excites the motor neurons in our eyes.

Serotonin and norepinephrine levels drop and acetylcholine increases. ACI stimulates diffuse areas of brain in unpredictable manner.

\subsection*{Content of Dreams}
Everyone dreams. Dreams are experienced as if they were real. They tend to have coherent but bizarre storylines. Usually they have mundae, everyday content; roughly half of the content is linked to recent experiences. Dreams can be influenced by external stimuli.

Most of us dream in colour.

Dreams are fleeting: they must be remembered immediately upon waking or they will be forgotten.

\subsection*{Theories about the Function of Dreams}
There are several different theories as to the functions of dreams:
\begin{itemize}
\item {\bf Wish Fulfillment Psychoanalytical Theory}: Dreams provide a psychic safety valve and often express otherwise unacceptable feelings. They contain manifest remembered content and a latent hidden meaning.
\item {\bf Information Processing}: Dreams help us sort out the day's events and consolidate our memories.
\item {\bf Psychological Function}: Regular brain stimulation from REM sleep may help develop and preserve neural pathways.
\item {\bf Activation Synthesis}: REM sleep triggers impulses that evoke random visual memories which our sleeping brain weaves into stories.
\item {\bf Cognitive Developmental Theory}: Dream content reflects the dreamers' cognitive development, i.e. his own knowledge and understanding.
\end{itemize}

Really, though, the only thing we know for sure is that REM sleep occurs in all mammals but no reptiles, decreases with age, and corresponds with the greatest period of neural development.

\section*{Vision}
Eyes \textbf{transduce} or transform light into neural messages that our brain can process.

The eye is comprised of the following systems:
\begin{enumerate}
\item The {\bf cornea} bends light to provide focus and also provides protection for the rest of the eye.
\item {\bf Pupils} are small, adjustable openings surrounded by the iris.
\item An {\bf iris} is a coloured muscle that adjusts light intake; it dilates or constricts in response to light intensity and to inner emotions.
\item Behind the pupil is the {\bf lens}, which focuses incoming light into an image on the retina.
\item The {\bf retina} is a multilayered tissue on the eyeball's sensitive inner surface.
\end{enumerate}

\subsection*{The Retina}
Light enteres the retina's outer layer of cells and proceeds to its buried receptor cells: the rods and cones. Light energy triggers chemical changes that would spark neural signals, activating neighbouring {\bf bipolar cells}.

Bipolar cells activate neighbouring {\bf ganglion cells}, which converge to form the optic nerve that carries info to your brain, where the thalamus distributes the info.  Where the optic nerve leaves the eye there are no receptor cells, creating a blind spot.

Cones cluster in and around the \textbf{fovea}, the retina's area of central focus. Many cones have their own hotline to the brain, bipolar cells that help relay the cone's individual message to the visual cortex, which devotes a large area to input from the fovea.

Since cones are better able to process fine detail, we tend to have much better vision in the centerr of our field of view than in our peripherals.

Cones enable perception of colour and become ineffectual in dim light. Rods enable black and white vision, but remain sensitive in dim light.

\subsection{Visual Information Processing}
At entry level, the retina processes information before routing it via the thalamus to the brain's cortex. The retina's neural layers, which are actually brain tissue which migrated to the eye during early fetal development, don't pass just along the electrical impulses, they also help to encode and analyze the sensory info.

After processing receptors and cones, information travels to your bipolar cels, then to ganglion, through the axons making up the optic nerve, to the brain.

Any given retinal area relays its info back to a corresponding location in the visual cortex, in the occipital lobe at the back of your brain.

The same sensitivity that enables retinal cells to fire messages can lead lead them to misfire as well.

\subsection*{Feature Detection}
Nobel prize winners David Hubel and Torsten Wiesel demonstrated that neurons in the occipital lobe's visual cortex receive information from individual ganglion cells in the retina. These \textbf{feature detector} cells derive their names from their ability to respond to a scene's specific features, to particular edges, lines, angles and movements.

Feature detectors in the visual cortex pass info to other cortical areas where teams of cells respond to more complex patterns.

\subsection*{Parallel Processing}
The brain divides a visual scene into subdimensions, such as colour, movement, form and depth. To recognize a face, brain integrates information the retina projects to several visual cortex areas, compares it, and enables images.

\textbf{Blindsight} is a localized area of blindness in part of some people's fields of vision. While these people could not see any sticks in these areas during an experiment, when asked to guess they routinely succeeded. There appears to be a second mind, a parallel processing system.

\section*{Colour Vision}
{\bf Thomas Young} and {\bf Hermann von Helmholtz} made the\textbf{trichromatic theory}, that the brain has three colour receptors: red, green and blue.

Colour-blind people lack functioning red or green sensitive cones, sometimes both. If this is true, how can those blind to red and green see yellow? {\bf Ewald Hering}, a physiologist, found a clue in the occurrence of \textit{afterimages}.

When you stare at a green square and then white, you see red. Stare at a yellow square and you will see later see blue. Hering determined that there must be two additional colour processes, one responsible for red versus green and one for blue versus yellow.

This was confirmed as \textbf{opponent processing theory}. As visual info leaves the receptor cells, we analyze it in terms of three sets of opponent colours: \textit{red-green, yellow-blue, white-black.}

In the retina and the thalamus, where impulses from the retina are relayed en route to the visual cortex, neurons are turned on by red but turned off by green.

When we stare at a white after green, only the red part of the green-red pairing fires normally, the green part is tired.

Colour processing occurrs in two stages, the retina's red, green and blue cones respond in varying degrees to different color stimuli, as the Young-Helmholtz trichromatic theory suggested.

Their signals are then processed by the nervous system's opponent-process cells, en route to the visual cortex.

\section*{Other Senses}
\subsection*{Touch}
Skin sensations are variations of the basic four:
\begin{enumerate}
\item Pressure: such as stroking pressure spots to tickle
\item Warmth: stimulating cold and warm spots produces the sensation of hot
\item Cold: touching adjacent cold triggers a sense of wetness, like cold metal
\item Pain: repeated stroking of a pain spot creates an itch
\end{enumerate}

\textbf{Kinesthesis} is a person's sense of position and movement of his or her body parts.

The \textbf{vestibular sense} monitors one's head's and thus one's body's position and movement. The biological gyroscopes for this sense of equilibrium reside in one's inner ear. The semicircular canals and the \textit{vestibular sacs} which connect the canals with the cochlea contain fluid that moves when one rotates their head or tilts.

The movement stimulates hair-like receptors which send back messages to the cerebellum at the back of the brain, enabling one to maintain body position and balance.

If one spins around, neither the fluid in the canals or the kinesthetic receptors immediately return to their neural state. The dizzy aftereffect fools the brain with the sensation that one is spinning.

\subsection*{Pain}
No one type of stimulus triggers pain, instead there are different \textit{nociceptors} detect hurtful temperatures, pressure or chemicals.

{\bf Ronald Melzack} and biologist {\bf Patrick Wall}'s \textbf{gate-control theory} attempts to explain pain. The spinal cord contains small nerve fibers that conduct most pain signals and larger fibers that conduct most other sensory signals.

They theorized that the spinal cord contains a neurological gate. When tissue is injured, the small fibers activate and open the gate and you feel pain. Large-fiber activity closes the gate, blocking pain signals and preventing them from reaching the brain.

Therefore one way to block pain signals is to stimulate by massage, etc, gate closing activity in the large neural fibers.

When we are distracted from pain, and soothed by endorphins, our experience of pain may be greatly diminished.

\subsubsection*{Psychological Influences}
People tend to record pain's peak moment, and how much pain they felt at the end. We tend to perceive more pain when others also seem to experience it.

\subsection*{Taste}
There are five taste receptors:
\begin{enumerate}
\item sweet
\item sour
\item salty
\item bitter
\item umami - the flavor enhancer monosodium glutamage
\end{enumerate}

\subsubsection*{Taste Aversion}
Nauseation can be paired with a food and it quickly becomes inedible. This type of learning, {\bf cognitive learning}, refers to acquiring new behaviours and information mentally rather than by direct experience.

\subsection*{Smell}
We smell when molecules of a substance carried in the air reach a tiny cluster of 5 million or more receptor cells at the top of each nasal cavity. 350 or so are receptor proteins that recognize each particular odor molecules.

Smell can evoke feelings and memories: a hot-line runs between the brain area receiving info from the nose and the brain's ancient limbic centers associated with memory and emotion.

\section*{Perceptual Organization}
German psychologists noticed that when given a cluster of sensations, people organize them into a \textbf{gestalt}, a form or whole.

The whole may exceed the sum of its parts, our brain makes inferences.

\subsection*{Form Perception}
The \textbf{figure-ground} is the organization of the visual field into objects, the figures, that stand out from their surroundings, the ground.

\subsection*{Grouping}
To bring order and form to the sensations of color, movement, light and dark contrast, our minds follow certain rules for grouping stimuli.

\begin{enumerate}
\item {\bf Proximity}: We group nearby figures together
\item {\bf Similarity}: We group simiilar figures together
\item {\bf Continuity}: We perceive smooth, continuous patterns rather than discontinuous ones
\item {\bf Connectedness}: We perceive objects overlayed as one unit
\end{enumerate}

We also subconsiously assume {\bf closure} by filling in gaps to create a complete, whole object.

\section*{Depth Perception}
The ability to see objects in three dimensions enables us to estimate distance. Infants innately know to not approach a \textbf{visual cliff}.

\subsection*{Binocular Cues}
Two eyes are better able to judge distance than one. Because of their separation, their \textbf{retinal disparity} provides an important binocular cue to the relative distance of different objects.

\subsection*{Monocular Cues}
When looking straight ahead, these cues help us tell whether a person is close or far away. Influence our everyday perceptions.

\subsection*{Motion Perception}
We perceive motion as "shrinking objects are retreating" and "enlarging objects are approaching". The brain perceives a continuous movement in a rapid series of slightly varying images, called \textit{stroboscopic movement}.

Marquees and holiday lights create another illusion of movement using the \textbf{phi phenomenon}. When two adjacent stationary lights blink on and off in quick succession, we perceive a single light moving back and forth between them.

\subsection*{Perceptual Constancy}
The ability to recognize objects without being deceived by changes in their shape, size, brightness, or color is an ability called \textbf{perceptual constancy}.

Shape constancy is the ability to form familiar objects as constant even while our retinal image of it changes. Size constancy is the ability to perceive objects as having a constant size in the same way.

\subsection*{Lightness Constancy}
We perceive an object as having a constant lightness even while its illumination varies. Perceived lightness depends on relative luminance, the amount of light an object reflects relative to its surroundings.

\subsection*{Color Constancy}
If you view only part of a red apple, its color will seem to change with light. If you view the whole apple, its color will remain roughly the same, a phenomenon called color constancy.

\section*{Perceptual Interpretation}
\subsection*{Sensory Deprivation and Restored Vision}
Writing to John Locke, {\bf William Molyneux} wondered whether "a man born blind, and now adult, taught his touch to distingiuish between a cube and a sphere, could, if made to see, visually distinguish the two. Locke's answer was no, as the man would never have learned the difference.

\subsection*{Perceptual Adaptation}
Our \textbf{perceptual adaptation} to changed visual input makes the world seem normal again. If we saw everything upside down, we would adapt and consider this the norm.

\subsection*{Perceptual Set}
Our expectations give us a mental predisposition that greatly influences what we perceive. People perceive an adult-child pair as looking more alike when they are parent and child.

Once we have formed a wrong idea about reality, we have more difficulty seeing the truth. Perceptual sets include not liking something when told it will tasta awful, versus if it was said to be expensive and of high quality.

\subsection*{Context Effects}
Imagine hearing a noise interrupted by the words "eel on a wagon" you would perceive the first word to be wheel. This phenomenon suggests that the brain can work backward in time to allow a later stimulus to determine how we perceive an earlier one.

\subsection*{Emotion and Motivation}
Perceptions are influenced not only by our expectations and by the context, but also by emotions. Walking destinations look farther away to those who have been fatigued. Hills look steeper to those wearing a backpack or listening to sad music.

Motives also matter: if seeing one outcome results in reward, it is more likely to be seen.

\subsection*{Perception and the Human Factor}
\textbf{Human factor psychologists} determine whether designs make sense to the average person.

Curse of knowledge: When you know a thing, it is very difficult to mentally stimulate what it is like to not know it.

\section*{Experimentation}
\subsection*{Ivan Pavlov}
{\bf Ivan Pavlov} was a Russian physiologist, obsessed with his research, who had an assembly line of experiments ready to go. He invented a surgical procedure by which you could access an animal's stomach, drain fluid, and watch digestion.

He wanted to see if different foods would cause the animal to secrete different fluids. But as soon as food was presented, the dog would salivate and ruin the secretions. He needed the dog not to do this.

He began classical or \textbf{Pavlovial} conditioning. When presented with some stimulus, a person becomes aroused. Pairing this with something not arousing, a person still becomes aroused. Now remove the stimulus, leaving just the originally not arousing object, and a person still becomes aroused, if less strongly.

The \textbf{unconditioned stimulus (US)} is stimulating regardless of the situation, intrinsically having the existing stimulus response association.

At this stage, the non-stimulating object is a \textbf{neutral stimulus (NS)}.

When the US is paired with the NS, we get a \textbf{conditioned stimulus}. This would not normally result in stimulus, but now that the NS and US have been paired, it does.

In his dog experiment, Pavlov tried things such as using a bell (NS) plus food (US) creating a response, salivation.

Afterwards, just the bell now classically/conditioned stimulus causes the dog to salivate.

\textbf{Extinction} occurs when the CS no longer stimulates, although it can much more easily be reimplemented with \textbf{spontaneous reconditioning}.

This condition was also present in drug consumption resulting in lower endorphine levels. After repeated uses of drugs, the NS routine, becomes associated with heroin (US), resulting in an unconditioned result (UR) of lower endorphins.

Eventually, the routine, now a CS, requires more heroin to get the same result.

\subsubsection*{Spontaneous Recovery}
After a CR has been conditioned, extinguished, and then followed by a rest period, presenting the CR again may lead to a return of the conditioned response despite the lack of further conditioning.

\subsubsection*{Generalization and Discrimination}
Ivan Pavlov conditioned dogs to drool when rubbed, and drooled when scratched. Ivan Pavlov conditioned dogs to drool at bells of a certain pitch. Generalization refers to the tendency to have conditioned responses triggered by related stimuli.

Discrimination refers to the learned ability to only respond to a specific stimuli, preventing generalization.

\subsection*{John Watson}
John Watson designed the {\it Little Albert} experiment:

Watson gives Albert a little white rat, he likes it. Watson then creeps up behind him and beats a gong. He gets upset, crying, and fearful. After conditioning, the white rat results in Albert's fear.

\subsection*{Edward Thorndike}
\textit{The Law of Effect}: Thorndike places cats in a puzzle box, they were rewarded with food and freedom when they solved a puzzle.

Thorndike noted that the cats took less time to escape after repeated trials and rewards.

The \textbf{law of effect} states behaviours followed by a favourable consequences become more likely, and behaviours followed by unfavourable consequences become less likely.

\subsection*{B.F. Skinner}
Skinner was once the most widely-cited, famous psychologist. He saw potential for exploring and using Edward Thorndike's principles much more broadly. One of his experiments was as follows:

He puts a rat in a cage with the ability to dispense food using a foot pedal. How does the ability of the rat to get food by performing this action affect it's efficiency at pushing the button?

\subsubsection*{Behavioural Contigencies}
Positive reinforcement: supply a positive consequence after a good behaviour. Supplying a negative outcome, punishment, decreases the preceding behaviour.

For a bad consequence, we supply punishment or remove negative reinforcement. Negative reinforcement is not punishment, it is the removal of a negative (such as punching someone until they say stop).

\subsubsection*{Reinforced Schedules}
Reinforced schedules can influence strength of repetition of behaviours by the provided stimulus. Two kinds of reinforcement: negative and positive, which includes the release of food pellets every time an animal does something properly, etc.

It is a fixed ratio if happens every time at a set duration. Behaviour changes as a result of this.

Slot machines were based on B.F. Skinner and his behavioural conditional. If you are rewarded on an unpredicatble, random variable ratio, you will continue the activity for a much longer period of time.

We can also have an interval contingency. Ex: something happens every 20 seconds regardless of number of actions in between.

Variable intervals cause us to have less urgency with which we perform any actions.

Types of Reinforcement Schedules:
\begin{enumerate}
\item Ratio: Fixed
\item Ratio: Variable
\item Interval: Fixed
\item Interval: Variable
\end{enumerate}

We get ritualistic, superstitious behaviour in order to get food expecting it to be the proper way to do so. For example: if a chicken dances then presses a button which releases food, it is apt to dance each time before it pushes the button.

\subsubsection*{Parental Use of Punishment}
\begin{enumerate}
\item Apply an adverse consequence (not physical!) or remove a desired conseqence
\item Immediate
\item Just strong enough
\item Consistent and predictable
\item Punish the behaviour - not the child
\item Replace with desired alternative
\end{enumerate}

\subsection*{John Garcia}
John Garcia was a behaviourist who named the {\bf Garcia Effect}. He believed classical conditioning may involve thinking on the part of the organism.

He puts the focus back on the thinking organism in the model: he takes a group of rats and shocks them.

He presented the rats with some conditioning thing: a sight, sound, or taste, as weall as a adverse effect (radiation poisoning). He was presented with two major conclusions:

\begin{itemize}
\item The rats would develop an aversion to the taste which caused them to be sick, even if that sickness only manifested hours later.
\item The rats would not develop aversion to the sights or sounds, only to the tastes.
\end{itemize}

\section*{Memory}
We can only keep around seven things in our memory at once, though we can expand this number through chunking (e.g. combining multiple pieces of information into a single bit) our data.

\subsection*{Types of Memory}
The {\bf Stage Model of Memory} gives us three different types of memory:
\begin{itemize}
\item {\bf Sensory Memory} is the temporary storage of sensory information. It has a high capacity, but lasts less than a few seconds.
\item {\bf Short-term Memory} is the brief storage of information currently being used. It has a limited capacity, and lasts under half a minute.
\item {\bf Long-term Memory} is the relatively permanent storage of information. It has a potentially unlimited capacity, and can potentially last forever (or at least for a very long time).
\end{itemize}

Information which passes through a "trigger" is transferred to the STM. It takes elaborate rehearsal to transfer information to the LTM, generally by subjecting information to deeper processing.

\subsection*{Forgetting}
Forgetting can occur at any memory stage: as we process informaiton, we filter, alter, or lose much of it.

\begin{center}
{\bf Sensory Memory} \\
The senses momentarily register an amazing detail. \\
$\downarrow$\\
{\bf Working Memory / STM} \\
A few items are both noticed and encoded. \\
$\downarrow$\\
{\bf LTM} \\
Some items are altered or lost. \\
$\downarrow$\\
{\bf Retrieval from LTM} \\
Depending on interference, retrieval cues, moods, and motives, some things get retreived and some don't.
\end{center}

It is interesting to note that {\bf deja vu} is caused by our subconscious and conscious streams being processed at different speeds: in essence, making us process something, then forget that we've done so and process it {\it again} subconsciously.

\section*{Intelligence Testing}
Intelligence tests are a series of questions and other exercises which attempt to assess people's mental abilities in a way that generates a numerical score, so that one person can be compared to another. Intelligence, then, can be defined as {\it whatever intelligence tests measure}.

{\bf Francis Galton} was a British Renaissance mathematician who believed that intelligence was a biological definition: it was determined by heredity and necessitated a powerful brain and nervous system. He was an ardent supporter of {\bf eugenics} for a long period, until his studies disproved his theories.

{\bf Alfred Binet} was a Frenchman of no special background who became interested in psychology around the same time as Galton. According to him, psychology was not fixed but instead grew naturally with experience. It is a loose colleciton of various capabilities such as memory, attention, reasoning... all tied together by good judgement.

He believed that intelligence was {\it "the tendecy to take and maintain a definite direciton, the capacity to make adaptations for the purpose of attaining a desired end, and the power of auto-criticism"}.

His first test was developed around 1905. This test involved testing children on various tasks by age in order to determine at which age the "normal" Person could accomplish certin tasks. these included the following:
\begin{itemize}
\item Two-year-old
\begin{itemize}
\item Follow follow a lighted match with their eyes (attention)
\item Unwrap and eat a piece of candy (simple tasks)
\item Handshakes, etc (cultural)
\end{itemize}
\item Six-year-old
\begin{itemize}
\item Explain the differences between object
\end{itemize}
\item Ten-year-old
\begin{itemize}
\item Rhyme two words
\item Reverse clock hands
\item Construct a complex sentence
\end{itemize}
\end{itemize}

We could score his test by determining the oldest age at which their test scores were perfect, then adding one-fifth of a year for each further correct answer until all were wrong in a year. He decided that the difference between one's physical and metnal age should be below two years for a person to be considered normal; children with mental ages more than two years below their physical age were called retarded.

Note that Binet didn't consider "retarded" to be a permanent state: he even created exercises esigned to bring children up to standard levels. He even worried that this test score would become a "brutal number".

{\bf Charles Spearman} was a British mathematician who did a lot of work with statistics, much like Galton. He worked heavily in the field of correlation analysis and factor serparation. He began his work in psychology by using his correlation methods on the standard intelligence test of his time, and came to the following conclusion:

There exist two types of intelligence testing items, each testing different things. The {\bf "g" factor} was general intelligence: an overall mental ability common to all intellectual tasks. this seemed to be biological, or inherited. The {\bf "s" factors} were specific to unique intellectual tasks: an environmental or learned type of intelligence. Note that the "g" factor supports Galton's work, and the "s" factor supports Binet's work.

This gave us the ability to determine the mechanical, spacial, verbal, and numerical intelligence scores, where general intelligence was the intersection of them.

{\bf William Stern} defined the {\bf Intelligence Quotient} \[ IQ = \frac{Mental\ Age}{Chronological\ Age} * 100 \] This creates today's basis: an IQ of 100 is considered average.

{\bf Lewis Terman} was the first American psychologist to get a copy of Binet's intelligence test after it had gained worldwide acclaim. He translated it and renamed it to the Stanford-Binet test, which became amongst the most famous tests in history.

Terman based eugenic conclusions upon a single test of two Mexican and two Native American children in Arizona which used a non-native language and non-culturally relevant examples, i.e. he used Binet's instrument to prove Dalton's theory.

{\bf David Wechsler} was an American who created a few intelligence scales shortly after this which are still used today (adapted and translated).

Wechsler defined intelligence as the {\it "aggregate or global capacity of the individual to act purposely, rationally, and with awareness of their environment"}. His tests consist of awareness, meaningfulness, rationality, and worth (in terms of a person's self-analysis of thought processes). {\it Intelligent behaviour is defined by its context}.

\subsection*{Modern Uses for IQ Testing}
IQ tests are now used as estimates of general intelligence. They identify specific assets and liabilites in specific fields (e.g. the need for advanced education) or to identify mental deficiencies or dysfunctions.

Note that the IQ is no longer a quotient, it is a normalized score centered around 100.

\section*{Motivation}
There are many ways psychologists try to measure one's motivation:
\begin{itemize}
\item Intensity
\begin{itemize}
\item Energy, enthusiasm, degree of effort
\item Measured by perceived level of effort or by psychological arousal
\end{itemize}
\item Direction / Choices
\begin{itemize}
\item Wolitional approach or avoidance of alternative activities
\item Selection of outcomes worthy of effort
\item Degree of task difficulty selected
\end{itemize}
\item Persistence
\begin{itemize}
\item Commitment to choices
\item Continued effort following frustration (goal-blocking)
\end{itemize}
\end{itemize}

\subsection*{Mechanistic Approach}
The {\bf Mechanistic} approach is to theorize that one's motivations come from within -- our instincts. We define {\bf instincts} as very rigid and predictable behaviours which arise from a species under certain circumstances; in addition, instincts are demonstrated by all members of a species.

Another common theory assumes that the {\bf needs and drives} within a person are what motivates him. {\bf Clark Hull}, {\bf Konrad Lorenz}, and {\bf Sigmund Freud} were all main proponents of this theory.

Hull was very much in favor of the science of psychology; he was driven to explain psychology in math-like terms. Lorenz spent a great deal of time studying animal behaviour and, in turn, projected these behaviours onto people. He believed he had a need to be aggressive in the same way that we have a need for food, and equated aggression satisfying with hunger saiting. Freud insisted that intrapsychich conflicts are the internal needs and wishes which we attempt to sate. He also believed that much behaviour was due to over-compensating (e.g. having an orgy with four lesbians to fuck away the gay).

A {\bf drive} is an aroused state or need which needs some sort of attention to fulfil. {\bf Drive-Reduction Theory}, then, refers to the idea that humans are motivated to reduce these drives and restore ourselves to homeostatis.

The {\bf Incentive Value} theory states that the motivations for one's behaviours come not from within, but from the available outcomes. In effect, this is a learning model: as we receive positive outcomes, the behaviours which lead to these situations are reinforced.

\subsection*{Humanistic Approach}
The {\bf Humanistic} approach (heralded by {\bf Abraham Maslow} and {\bf Carl Rogers}) argues that humans are on a lifelong journey to become more: to become better, stronger, more complete. Maslow propsed the hierarch of needs and motivations which declares that each potential need has precedence over others. Note that there has been no research which supports this theory.

Rogers, on the other hand, created the {\bf Discrepency Theory}, which deals with the commonalities between our real and ideal selves. There exist two different categories of selves:
\begin{itemize}
\item {\bf Incongruent Selves} are those which have a large divide between their real and ideal selves. This can be demoralizing, and can make self-actualization difficult.
\item {\bf Congruent Selves} are those with a smaller discrepency: self-actualization is more attainable and thus this tends to be more motivating.
\end{itemize}

\subsection*{Expectancy Approach}
The {\bf Expectancy} approach was invented by {\bf Henry Murray} and {\bf Albert Bandura}. Murray declared that humans were rationalizers who were always searching for the "best deal", the most effective outcome. He believed motivation to be the product of the outcome expectations and the incentive value of the outcome.

Bandura added on to this; he noted that this model did not accurately predict human behaviour for a variety of situations. He added {\it efficacy expectations} to the model, to be calculated prior to the outcome expectations: these are a person's beliefs in themselves, i.e. their belief in their ability to perform whatever behaviour which would lead to one of Murray's outcome expectations.

In essence then, we have the following: efficacy expectations are based on one's belief that they have the skills, experience, strategy, equipment, etc to perform as well as the belief that one would actually be able to execute these actions under the given circumstances; outcome expectations are based on the belief that particular behaviours will actually produce certain outcomes.

\subsubsection*{Sources of Belief in Self-Efficacy}
There are multiple sources of our confidence in our own efficacy:
\begin{itemize}
\item Performance and Accomplishments
\begin{itemize}
\item Mastery experiences
\item History of success / failure
\item Most influencial long-lasting source (essential)
\end{itemize}
\item Vicarious Experiences
\begin{itemize}
\item Observing success / failure of others
\item Social learning
\item Only influencial until reproduction
\end{itemize}
\item Social Persuasion
\begin{itemize}
\item Encouragement from others (usually verbal)
\item Dependant on source credibility
\item Brief duration
\end{itemize}
\item Emotional Arousal
\begin{itemize}
\item Pysched up / pysched out
\item Brief duration
\end{itemize}
\end{itemize}

\subsection*{Cognitive Approach}
{\bf Edward Deci} and {\bf Richard Ryan} declared that {\bf Cognitive Evaluation Theory} shows that there is both {\bf intrinsic} and {\bf extrinsic} value for any given outcome: e.g. the performance is rewarding vs the outcome of the performance is rewarding. From this we get the "means to an end" or an "end itself", and pleasure either in the process or the outcome. Intrinsic values are not necessarily goal-oriented, but extrinsic values must be.

We also tend to associate intrinsic value with play and extrinsic value with work.

Intrinsic and extrinsic values can be either complementary (additive) or adversary (destructive). When they are additive, we are motivated to produce maximum results so as to test skills and efficacy. Otherwise, we are simply motivated to produce the minimum performance necessary to produce extrinsic rewards.

\subsection*{Social Approach}
{\bf Henri Tajfel} and {\bf John Turner} worked together to produce the {\bf Social Identity Theory}. Basically, we have two sources of esteem: personal idenity and social identity.
\begin{itemize}
\item {\bf Personal Identity} is individual esteem: the ability to distinguish oneself from others, enhanced by achievements. This deals with accomplishments, competence, skills, et cetera.
\item {\bf Social Identity} is collective esteem: the ability to align oneself with others, enhanced by affiliations. This deals with relationships, teams, groups, et cetera.
\end{itemize}

\section*{Conversion Hypothesis}
Conversion hypothesis is the trend of middle-aged women to present physiological symptoms which could not be traced to a physical root; i.e. hypochondria. {\bf Sigmund Freud} became interested in these people during his carreer and his interactions with these women formed the basis of the bulk of his theories.

Freud also gathered a large amount of methods for analyzing his patients: hypnosis, word association, cocaine, dreams, et cetera.

According to Freud, consciousness is comprised of three distinct entities: the {\bf id}, which is responsible for our base desires; the {\bf ego}, which is the conscious, thinking part; and the {\bf superego}, which is responsible for our sense of outward morality.

Freud also believed in three levels to our personality: the {\bf conscious}, which is the self-aware area of our minds; the {\bf preconscious}, which is easily accessible mental material; and the {\bf subconscious}, which is a resevoir of thoughts and feelings kept hidden from our conscious because they are deemed unacceptable.

He equated these levels to an iceberg: though only the conscious is readily apparent, the other layers exert significant pressure and influence our daily lives. Even accidental word slips would tend to be caused by subconscious desires.

{\bf Rorscharch Tests} are tests which provide a random stimuli such as a splatter of ink in order to attempt to determine which subconscious thoughts are floating around one's mind based on their identification of the ink blot.

{\bf Thematic Apperception Tests} are based on the same principle: real pictures were presented and people are asked to provide the backstories. Somehow, this measures the amount of "achievement motive" people have, based on how high-reaching the backstories are.

\subsection*{Psychosexual Stages of Personality Development}
\begin{itemize}
\item Oral (0-18 months): pleasure centres in the mouth
\begin{itemize}
\item Erogenous zone: mouth
\item Developmental task: weaning
\item Oral Dependant personality: excessive gratification, passive, dependant, gullible
\item Oral Aggressive personality: excessive frustration, argumentative, cynical, exploitive, cruel, sarcastic
\item Expressed by gum chewing, mail-biting, smoking, kissing, eating disorders, alcoholism
\end{itemize}
\item Anal (18-36 months): pleasure focuses on bowel and bladdar elimination
\begin{itemize}
\item Erogenous zone: anus
\item Developmental task: toilet training
\item Anal Expulsive personality: excessive gratification, sloppy, careless, messy, disorderly, creative
\item Anal Retentive personality: excessive frustration, OCD, clean, rigid
\end{itemize}
\item Phallic (3-6 years): pleasure zone in the genitals, coping with incentuous feelings
\begin{itemize}
\item Erogenous zone: genitals
\item Developmental Task: sexual identification
\item Boys' Oedipal Complex: incestuous attraction to mother, castration anxiety, align with father to prevent being castrated and to gain mother's sexual affections
\item Girls' Oedipal Complex: at first, Freud's theory for girls was the opposite of that of the boys.
\item Girls' Electra Complex: penis envy, incestuous attraction to father {\it so she can gain a penis and become a complete person}.
\item Expressed by overly macho men, overly frilly (daddy issues) or dominant (masculine) girls, any sexual identity issues, expecially homosexuality
\end{itemize}
\item Latency (6 years - puberty): dominant sexual feelings
\begin{itemize}
\item Erogenous zone: none, quiet stage
\item Developmental task: solidifying gender roles
\end{itemize}
\item Genital (Puberty+): masturbation and sexual interests
\begin{itemize}
\item Erogenous zone: genitals for sex
\item Developmental task: intimacy and procreation
\item Balanced personality: caring, responsible, mutual gratification
\item Neurosis or Anxietal personality: repression of unresolved issues, intimacy issues, sexual disfunctions
\item Expressed by defence mechanisms, intimacy issues, sexual disfunctions
\end{itemize}
\end{itemize}

All of Freud's theories were once prevalent, but soon competing theories arose. The medical profession as a whole was an ardent supporter of psycho-therapy for a long time and thus supported Freud, but eventually moved toward neurology and more scientific approaches. Insurance companies were also quite against the costs associated with hiring psychologists vs prescribing pills. The legal profession as a whole also turned against Freud's theories: at first, only psychodynamic psychologist's were accepted as expert witnesses, but soon people realised that most psychologists would declare that any women under their care had probably been sexually assulted... which caused absurd amounts of illegitimate trials and people being sent to jail. Finally, Freud's theories were also assaulted scientifically by people who insisted Freud's theories were bad science (e.g. not falsifiable, based on a small sample of similar patients, post facto explanation, patently absurd...)

\section*{State vs Trait}
An individual's characteristic patterns of thoughts, feelings, and behaviours which persist over time tend to be the intersection between state and trait aspects.

{\bf State} shyness is when one's environment triggers shyness, e.g. talking to a girl, public speaking, etc. {\bf Trait} shyness is one's inherent shyness with regards to certain things, e.g. talking to people, being outside.

Most of the population exhibits state shyness, i.e. are shy in certain situations only. This results in minimal negative consequences. However, less than a quarter of people are trait shy, i.e. shy in all situations. This tends to have more severe consequences and will affect one's behaviour very much.

\section*{Dating}
Most dating tends toward the "sex for success" trade-off: evolutionarily, the male would bring protection, food, survival, et cetera, and the female would bring child-bearing, fertility, and mothering.

We must do three things to fulfill the biological imperative:
\begin{enumerate}
\item Survive until sexual maturity. One tends to be more selfish and self-centered before this point, as one must survive until the appropriate age.
\item Procreate.
\item Get offspring to sexual maturity. Parents tend to be less selfish and more protective.
\end{enumerate}

\subsection*{Optimal Strategy for Women}
A woman has only one opportunity per month to fulfill her imperative, over the course of around forty years. The optimal strategy for women, then, is to have a very selective strategy: she should not "waste" her opportunity without being assured that the potential mate is a worthy one.

\subsection*{Optimal Strategy for Men}
A man has millions of opportunities per month, over his entire life. Thus, the best strategy for men is to procreate at every opportunity.

\subsection*{Realistic Strategies}
Since women tend towards selectivity, they can thus demand restrictions such as monogamy from men. As such, men will look for those women which will give them the best chance of offspring, i.e. those that are youngest.

\subsection*{Beauty}
Beautiful people tend toward better opportunities: more dating prospects, better career paths, less jail time, more trusting. On the other side of this, beautiful people tend to have negative stereotypes associated with them, can have self-esteem dilemmas, and realise that their advantage has an expiry date.

\end{document}
