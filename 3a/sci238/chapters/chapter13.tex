\section{Chapter 13 -- White Dwarfs, Neutron Stars, Black Holes}
To scientists, dead stars are ideal laboratories for testing the most extreme theories of general relativity and quantum theory.

\subsection{White Dwarfs}
\subsubsection{What is a white dwarf?}
A white dwarf is essentially the exposed core of a low-mass star that has died and shed its outer layers in a planetary nebula. It is quite hot when it first forms (it was the inside of a star) but it slowly cools with time. White Dwarfs have masses like those of stars but sizes like that of Earth which is why they are generally quite dim compared to stars like the sun. THe hottest white dwarfs can shine brightly in high-energy light such as ultraviolet and X-rays.

A white dwarf's combination of starlike mass and a small size makes gravity near its surface very strong. Because there is no fusion to maintain heat and pressure, \bf{degeneracy pressure} combats the gravitational force. The same pressure supports brown dwarfs, it arises when particles are packed as closely as the laws of quantum mechanics allow. More specifically, in white dwarfs arises from electrons so it is called \bf{electron degneracy pressure}

\subsubsection{Composition, Density and Size}
{\bf Composition} of a White Dwarf reflects product's of stars final fusion stage. A white dwarf from something resembling our sun would consist mostly of carbon. (stars like the sun fuses helium into carbon in final stage of life). The \bf{bf denisty} of a white dwarf is so high that a teaspoon of its material would weight several tons. More massive white dwarfs are also smaller in size (the most massive being the smallest). The more massive a white dwarf is, the greater gravity compresses matter to a much greater density. Electrons in a white dwarf respond to compression by moving faster.

\subsubsection{The White Dwarf Limit}
The fact that electron speeds are higher in more massive white dwarfs leads to a fundamental limit on the maximum mass of a white dwarf. The \bf{White Dwarf limit} is \bf{$1.4M_Sun$} because anything larger than this would have electrons moving faster than the speed of light. Also called Chandrasekhar limit. In every observed case this limit holds true.

\subsubsection{White Dwarf in a binary System}
A white dwarf in a binary system can slowly gain mass if its companion is a main sequence of giant star. Matter coming from the other star forms a whirlpool like disk as it makes its way to the White Dwarf's surface (called an \bf{accretion disk}). In this way, a white dwarf can get Hydrogen.

\bf{Novae}: Hydrogen spilling towards the white dwarf heats up. If the temperature reaches 10 million K hydrogen fusion suddenly ignites. This thermonuclear flash causes the binary system to shine for a few weeks as a nova. (far less luminous than a supernova but can still shine as brightly as 100,000 suns). Accretion resumes after nova explosion subsides so process can repeat itself

\bf{White Dwarf SuperNovae}: Through repeating the previous process it is believed that a white dwarf gains mass. When it reaches the \bf{white dwarf limit} carbon fusion begins and explodes completely into what we call a \bf{White Dwarf Supernova}. This is however quite different from a \bf{massive star supernova}. Both shine with luminosities of 10 billion times that of the Sun but white dwarfs supernovas fade steadily and massive stars are more complicated. White Dwarf supernovas also lack hydrogen lines.

\subsection{Neutron Stars}
\subsubsection{What is a neutron star?}
A ball of neutrons created by the collapse of the iron core in a massive star supernova. Typically 10km in radius yet more massive than the sun. \bf{Neutron degeneracy pressure} supports neutron stars. The gravity on the surface makes the escape velocity about half the speed of light. Neutron stars spin rapidly when they are born and strong magnetic fields can direct beams of radiation that sweep through space.

\subsubsection{How were they discovered?}
First observational evidence 1967, radio waves at precise intervals (now refered to as \bf{pulsars}). Signal came from gaseous remains of supernova. It was a neutron star, pulsations arise becuase of conservation of angular momentum (rotation increases as size decreases). Neutron star's rotation slows over time. \bf{Pulsars} must be neutron stars because no other object could spine that quickly without tearing itself apart. White Dwarf 1/sec. Pulsar as fast as 625/second.

\subsubsection{Neutron Star in a binary System}
Like white dwarfs, neutron stars can burst back to life. Due to stronger gravity, the accretion disk on a neutron star is much hotter and denser than a white dwarf's. High temperatures in inner regions of the disk make it radiate powerfully in x-rays. Due to this emission these are often called \bf{X-ray binaries} and hundreds have been detected in the Milky Way. Pulsars of X-ray binaries accelerate with time, some rotating every few thousandths of a second. (called \bf{millisecond pulsars}).

Helium fusion can happen at a layer of the disk builds to 100 million K. Helium fuses rapidly and generates an \bf{X-ray burster} which lasts a few seconds and flares every few hours to every few days. Energy relased is 100,000 times more powerful than sun output, all in X-rays. After burst, accretion resumes.

\subsection{Black Holes}
\subsubsection{What is a Black Hole?}
A black hole is so compact that it has an escape velocity greater than the speed of light, neither light nor anything else can escape from within a black hole. They are actually spherical and not funnel shaped.

\bf{The Event Horizon}: The boundary between the inside of a black hole and the universe outside is called the event horizon. It marks the point of no return for objects, the boundary at which escape velocity equals the speed of light. Gets the name because we have no hope of learning about events that occur within it.

The \bf{size} of a black hole is usually the size of its event horizon, defined by the \bf{Schwarzschild radius}. Black hole with mass of the Sun has a Schwarzschild radius of about 3km. More massive black holes have a larger Schwarzschild radius.

A collapsing stellar core becomes a black hole at the moment it shrinks to a size smaller than its Schwarzschild radius.

Schwarzschild radius: \[ R_s = \frac{2GM}{c^2} \]

\subsubsection{Singularity and Limits of Knowledge}
Because nothing can stop the crush of gravity in a black hole, we might expect all matter that forms a black hole must ultimately be crushed to an infinitely tiny area and dense point in the center called a \bf{singularity}

According to Einstein's theory from your point of view a friend would never cross the event horizon even though he would vanish from view due to redshifiting. You would not survive to cross the event horizon due to gravity, however with supermassive black holes, tidal forces are weaker so it would be possible to enter the event horizon.

\subsubsection{Formation of a Black Hole}

Most massive stars may not succeed in blowing away all upper layers in supernova. If enough mass falls back to neutron star, it could exceed neutron star limit (3M Sun). Gravity would exceed degeneracy pressure and core collapses again with no known force to keep it from collapsing into a black hole.

\subsubsection{Observational Evidence}
Gravity alters its surroundings. Compelling observational evidence comes from studying X-ray binaries, some may contain black holes instead of neutron stars. The trick to learn is by measuring mass.

\subsection{Origin of Gamma Ray Bursts}
By far the most powerful bursts of energy we observe in the universe. Some appear to come from extremely powerful supernova explosions. A supernova from a neutron star does not release enough energy, however a supernova that forms a black hole (\bf{hypernova}) might be powerful enough to explain it.

\subsection{Summary Notes}
\begin{itemize}
\item Observation evidence exists for white dwarfs and neutron stars, is strong for black holes
\item All three can have close stellar companions in which they can accrete matter.
\item Black holes are holes in the universe that strongly warp time and space around them. Nature of singularities beyond frontier of current understanding.
\end{itemize}
