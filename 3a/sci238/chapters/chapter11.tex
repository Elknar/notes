\section{Chapter 11 -- Surveying the Stars}
\subsection{Properties of Stars}
\subsubsection{Luminosity}
The \textbf{apparent brightness} of a star is how bright it appears to us, or the amount of power reaching us per unit area (units are watts per square meter). The \textbf{luminosity} of a start is is the total power that the star emits (units are watts). Apparent brightness follows a inverse square law to distance.
\begin{align*}
    \text{apparent brightness} = \frac{\text{luminosity}}{4\pi \times (\text{distance})^2}
\end{align*}

The most direct way to measure a star's distance it through measuring its stellar paralax. This is found by comparing a star's shift against its background over 6 months. We calculate its \textbf{paralax angle} \[ d = \frac{1}{p} \]
Where d is the distance to that star in light years and p is the paralax angle (Note: 1 arcsecond $\rightarrow$ 3.26 lightyears = 1 parsec)

We tend to measure luminosities as orders of magnitude of our sun's luminosity, called \textbf{apparent magnitude} instead of apparent brightness and \textbf{absolute magnitude} instead of luminosity. For every 5 magnitudes we have a brightness factor of 100. So a magnitude 1 star is 100 times brighter than a magnitude 6 star. We define the absolute magnitude as the apparent magnitude if owould have itf it were at a distance of 10 parsecs.

\subsubsection{Temperature}
The temperature of a star usually means its surface temperature since its the easiest to measure. A star's temperature is easier to measure than luminosity since it does not vary with distance. We can measure a star's temperature with reasonable accuracy by measuring its color. This is done by comparing its apparent brightness in two different colors of light (usually blue and red).

We run into problems when interstellar dust interferes with the color of a star so astronomers use a star's spectral lines instead. Stars showing spectral lines of ionized elements are fairly hot because it takes high temperature to ionize atoms. In contrast, stars displying spectral lines of molecules are relatively cool. Stars are classified by their \textbf{spectral type} OBAFGK in decreasing order of temperature. These are often divided farther using numbers. For example our sun is a G2 star meaning its hotter than a G3 but cooler than a G1.

\begin{tabular}{|l|l|l|l|l|}
\hline
Type & Example(s) & Temp. & Key Absorption Line Features & Brightest (color) \\
\hline
O & Stars of Orion’s Belt & $>$30k K & Strong ionized He, weak H & $>$97 nm (ultraviolet) \\
\hline
B & Rigel & 30k-10k K & Strong neutral He, some H & 97-290 nm (ultraviolet) \\
\hline
A & Sirius & 10k-7.5k K & Very strong H & 290-390 nm (violet) \\
\hline
F & Polaris & 7.5k-6k K & Some H, some ionized Ca & 390-480 nm (blue) \\
\hline
G & Sun, Alpha Cent. A & 6k-5k K & Weak H, strong ionized Ca & 480-580 nm(yellow) \\
\hline
K & Arcturus & 5k-3.5k K & some metals, some molecules & 580-830 nm (red) \\
\hline
M & Betelgeuse, Prox. Cent. & $<$3.5k K & Strong molecules & $>$ 830 nm (infrared) \\
\hline
\end{tabular}

\textbf{History of Spectral Types}
Spectral types were made at Harvard College by Edward Pickering's computers (women who'd studied physics or astronomy). The first was Williamina Flemming who classified A-O by the descending strength of hydrogen lines. Annie Jump Cannon modified this existing classification by reordering and removing classes until the OBAFGKM that is used today was left. Finally Cecilia Payne-Gaposchkin discovered that stars were all made of the same material and the lines reflected ionization levels which indicated surface temperature.

\subsection{Mass}
Measuring mass is very difficult and we can only really do it on binary star systems. We do this by applying Newton's version of Kepler's third law

Binary star types:
\begin{itemize}
\item \textbf{visual binary: } we can see each star distinctly, sometimes one star is too dim to see but we can observe the shift of the visible star
\item \textbf{eclipsing: }  a pair of stars that orbit in a plain of our line of sight, we rotate between seeing the combined light of both stars (no eclipse) and only the light of one star (full eclipse)
\item \textbf{spectroscopic: } we need to use Doppler shifts to detect its nature
\end{itemize}

\subsection{Patterns Among Stars}
These are ways of charting stars, the x-axis is the surface temperature (OBAFGKM) and the vertical axis is luminosity ($L_{sun}$) on a logarithmic scale. We can also infer the star's radius from the chart because a star's luminosity is based on its surface temperature and radius.
\begin{align*}
L &= 4\pi r^2 \times \sigma T^4 \\
\sigma &= 5.7 \times 10^{-8} W/(m^2 \times \text{Kelvin}^4)
\end{align*}
Where L is luminosity, r is radius, $\sigma$ is amount of thermal radiation emmited per unit area constant, and T is the star's temperature. This means that the radius of a star increases along a diagonal from lower left to upper right.

Stars cluster on the H-R diagram:
\begin{itemize}
\item \textbf{main sequence: }streak running from upper left to lower right
\item \textbf{supergiants: }upper right
\item \textbf{giants: }between supergiants and main sequence
\item \textbf{white dwarfs: }lower left
\end{itemize}

Like with spectral classes, astronomers assign luminosity classes describing the region of the H-R diagram that a star falls in. I is for super giants, III is for giants and V is for main sequence. II and IV are for those that fall inbetween. White dwarfs fall outside the classification and are called wd. Stars with higher luminosity have larger radii as well.

We combine spectral type and luminosity class together to identify stars.

\subsection{Main Sequence}
Main sequence stars are the majority of stars that we observe and because of that we have found more patterns within them. Mass increases as we go up the strip of main sequence stars on the H-R Diagram. We also see that low mass stars are much more common than high mass stars. Mass is the most important attribute of hydrogen fusing stars because it determines the balance point at which energy released by fusion equals the energy lost from the star's surface. This is what allows for the wide range of luminosities. Luminosity is very sensitive to mass (example a star 10 times as massive as the sun is 10000 times as luminous).

A luminous star must be very hot or very large. But a very small mass change is required to greatly increase the luminosity of a star, so their surface temperature must be much higher to account for this large increase in luminosity. This fits the H-R Diagram pattern of temperature increasing with luminosity. We can use the mass-luminosity-temperature relationship to estimate a star's mass just by knowing its spectral type.

A star is born with a set amount of hydrogen fuel, the amount of time that it can burn this fuel for is the star's \textbf{main sequence lifetime}. Lifetime is inversely proportional to the mass of the star. This is because as mass increases luminosity increases exponentially, so stars with higher masses may have more fuel but they burn it waaaay faster.

\subsection{Giants, Supergiants, and White Dwarfs}
\subsubsection{Giants and Supergiants}
These are much cooler but more luminous than the sun which tells us that they are huge. These stars have almost exhausted their hydrogen fuel supply and are trying not to collapse under their own weight. They do this by releasing fusion energy at a high rate which explains their high luminosity, and the need to radiate all this energy expands them to enormous size.

\subsubsection{White Dwarfs}
This is what happens when a giant runs out of fuel completely. The star ejects all of its outer layers and is left only with a dormant core. They are hot because they are still the core of a star, but dim because they have no way to radiate their energy.

\subsection{Star Clusters}
Stars are born from giant clouds and many stars can be born from the same cloud, so they tend to cluster.
\begin{itemize}
\item all stars in a cluster are about the same distance from earth
\item all stars in a cluster are about the same age
\end{itemize}

\subsubsection{Types of clusters}
\begin{itemize}
\item \textbf{open cluster: }found in disk of galaxy, young stars, up to several thousand stars, about 30 light years across
\item \textbf{globular cluster: }found in halo of galaxy: oldest stars, more than a million stars, 60-150 light years across
\end{itemize}

\subsubsection{Age of a Cluster}
We plot a cluster's stars on the H-R Diagram, and this tells us its age. For instance the Pleides open cluster has no stars of the O spectral class. This means that Pleidas is old enough that its O stars have finished their hydrogen fission and `died'. We call the point at which a cluster's main sequence diverges from the standard main sequence the \textbf{main sequence turnoff}. The age of the cluster is equal to the lifetime of stars at its main-sequence turnoff point (at its most massive star).
