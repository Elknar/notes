\subsection{Basic Astronomical Objects}
\begin{description}
\item[star] is a large, glowing ball of gas that generates heat and light through nuclear fusion in its core. Our Sun is a star.
\item[planet] is a moderately large object that orbits a star and shines primarily by reflecting light from its star. According to a definition
approved in 2006, an object can be considered a planet only if it (1) orbits a star; (2) is large enough for its own gravity to make
it round; and (3) has cleared most other objects from its orbital path. An object that meets the first two criteria but has not
cleared its orbital path, like Pluto, is designated a dwarf planet.
\item[moon (or satellite)] is an object that orbits a planet. The term satellite can refer to any object orbiting another object.
asteroid A relatively small and rocky object that orbits a star.
\item[comet] is a relatively small and ice-rich object that orbits a star.
\item[extrasolar planet] is a planet that orbits a star that is not our Sun.
\end{description}

\subsection{Collections of Astronomical Objects}
\begin{description}
\item[solar system] is the Sun and all the material that orbits it, including the planets, dwarf planets, and small solar system bodies. Although the term solar system technically refers only to our own star system (solar means “of the Sun”), it is often applied to other star systems as well.
\item[star system] is a star (sometimes more than one star) and any planets and other materials that orbit it.
\item[galaxy] is a great island of stars in space, containing from a few hundred million to a trillion or more stars, all held together by gravity and
orbiting a common center.
\item[cluster (or group) of galaxies] is a collection of galaxies bound together by gravity. Small collections (up to a few dozen galaxies)
are generally called groups, while larger collections are called clusters.
\item[supercluster] is a gigantic region of space where many individual galaxies and many groups and clusters of galaxies are packed more
closely together than elsewhere in the universe.
\item[universe (or cosmos)] are the sum total of all matter and energy -- that is, all galaxies and everything between them.
observable universe The portion of the entire universe that can be seen from Earth, at least in principle. The observable universe is
probably only a tiny portion of the entire universe.
\end{description}

\subsection{Astronomical Distance Units}
\begin{description}
\item[astronomical unit (AU)] is the average distance between Earth and the Sun, which is about 150 million kilometers. More technically,
1 AU is the length of the semimajor axis of Earth’s orbit.
\item[light-year] is the distance that light can travel in 1 year, which is about 9.46 trillion kilometers.
\end{description}

\subsection{Terms Relating to Motion}
\begin{description}
\item[rotation] is the spinning of an object around its axis. For example, Earth rotates once each day around its axis, which is an imaginary
line connecting the North Pole to the South Pole.
\item[orbit (revolution)] is the orbital motion of one object around another. For example, Earth orbits around the Sun once each year.
\item[expansion (of the universe)] is the increase in the average distance between galaxies as time progresses. Note that while the universe as a whole is expanding, individual galaxies and galaxy clusters do not expand.
\item[Doppler shift] light is \textbf{bluer} if the object is moving \textbf{towards} the Earth, it is \textbf{redder} if the object is moving \textbf{away} from the Earth. The intensity of the light is related to the speed at which the object is moving toward / away from the Earth.
\end{description}

\subsection{Telescopes}
\begin{description}
\item[Diffraction limit] The angular resolution before interference of light itself causes problems
\end{description}
