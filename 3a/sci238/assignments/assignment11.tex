\subsection{Assignment 11}

\textbf{ The Cosmic Distance Scale—Cepheids}
\begin{itemize}
    \item Cepheids with longer periods have higher luminosities
    \item How to use Cephids to measure distance:
    \begin{itemize}
        \item Step 1: Measure the period of the Cepheid's brightness variations.
        \item Step 2: Use the period-luminosity relation to determine the Cepheid's luminosity.
        \item Step 3: Calculate the Cepheid's distance from its luminosity and apparent brightness.
    \end{itemize}
\end{itemize}
\textbf{The Cosmic Distance Scale—From the Solar System to the Universe}
\begin{itemize}
    \item What baseline distance must we know before we can measure parallax?
    \begin{itemize}
        \item the Earth-Sun distance
    \end{itemize}
    \item Standard candle techniques
    \begin{itemize}
        \item white dwarf supernovae (distant standards)
        \item Cepheids
        \item main-sequence fitting
    \end{itemize}
\end{itemize}
\textbf{The Cosmic Distance Scale—Hubble's Law}
\begin{itemize}
    \item Hubble's law expresses a relationship between the distance of a galaxy and the speed at which it is moving away from us
    \item But before we can use Hubble's law, we must first calibrate it by measuring the distances to many distant galaxies with a standard candle technique
    \item meaning of Hubble's constant: It describes the expansion rate of the universe, with higher values meaning more rapid expansion
\end{itemize}
\textbf{Understanding Hubble’s Law}
\begin{itemize}
    \item Hubble’s law tells us that the more distant a galaxy is from Earth, the faster it is moving away from us
    \item more distant galaxies move at higher speeds
    \item a steeper slope (distance vs speed) for Hubble’s law would predict faster speeds for galaxies at particular distances
\end{itemize}
\textbf{Visual Activity: A Graph of Hubble’s Law}
\begin{itemize}
    \item galaxies with high speeds as measured from Earth are moving away from Earth and are farther from Earth than galaxies with lower speeds
    \item galaxies that have the lowest speeds are moving away from Earth and are closer to Earth than galaxies with high speeds
    \item galaxy B is twice as far from Earth as galaxy A. Hubble’s law predicts that galaxy B will be moving away from Earth with approximately twice the velocity of galaxy A
    \item he slope of Hubble’s law on the graph is actually steeper than that shown. In that case, the age of the universe would be younger than 14 billion years because the universe is expanding more rapidly than current data suggest
\end{itemize}

Which of these galaxies would you most likely find at the center of a large cluster of galaxies? a large elliptical galaxy

In which of these galaxies would you be least likely to find an ionization nebula? a large elliptical galaxy

If all the stars on the main sequence of a star cluster are typically only one-hundredth as bright as their main-sequence counterparts in the Hyades Cluster, then that cluster's distance is 10 times as far as the Hyades's distance.

Which of these galaxies is most likely to be oldest? a galaxy in the Local Group

About how many galaxies are there in a typical cluster of galaxies?  a few hundred

When the ultraviolet light from hot stars in very distant galaxies finally reaches us, it arrives at Earth in the form of visible light.

Why do virtually all the galaxies in the universe appear to be moving away from our own? Observers in all galaxies observe a similar phenomenon because of the universe's expansion.

If you observed the redshifts of galaxies at a given distance to be twice as large as they are now, then you would determine a value for Hubble's constant that is twice as large as its current value.

Redshift of value z: $1+z=\frac{\lambda_{obsv}}{\lambda_{emit}} = \frac{d_{now}}{d_{past}}$ where obsv is wavelength observed and emit is wavelength emitted and d is distance.

\textbf{Galaxy Formation—Spiral or Elliptical}
\begin{itemize}
    \item A collision strips gas out of a spiral galaxy, this tend to change the spiral galaxy into an elliptical galaxy because a galaxy cannot have a disk if it does not have gas
    \item High density tends to lead to more rapid star formation in a protogalactic cloud which leads to an elliptical galaxy, rather than a spiral galaxy because rapid star formation means that there may not be enough gas left to make a disk.
    \item High angular momentum leads to faster rotation which leads to a spiral galaxy, rather than an elliptical galaxy because faster rotation leads to collisions among gas particles that cause the gas to settle into a spinning disk, rather than a more spread out cloud.
\end{itemize}

Which of these items is a key assumption in our most successful models for galaxy formation? Some regions of the universe were slightly denser than others.

A collision between two large spiral galaxies is likely to produce a large elliptical galaxy.

The luminosity of a quasar is generated in a region the size of the solar system.

The primary source of a quasar's energy is gravitational potential energy.

Supermassive black holes found at the centers of galaxies are related to the properties of those galaxies in which of the following ways? The mass of the black hole is related to the mass of the galaxy’s bulge.

A collision and merger of two large elliptical galaxies will eventually produce a large elliptical galaxy.

Starburst galaxies are especially bright in infrared light.

The rate at which supernovae explode in a starburst galaxy that is forming stars 10 times faster than the Milky Way is about 10 times higher than in the Milky Way.



