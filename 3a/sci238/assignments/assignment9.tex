\section{Assignment 9}
Stages of birth of a star from first to last:\\
molecular cloud fragment, contracting cloud trapping infared light, protostar with jets, main-sequence star\\

From Highest to lowest temp:\\
main-sequence star, protostar with jets, contracting cloud trapping infared light, molecular cloud fragment\\

Fastest to slowest:\\
main-sequence star, protostar with jets, contracting cloud trapping infared light, molecular cloud fragment\\

Newly forming star has the greatest luminosity when it is a shrinking protostar with no internal fusion. Greatest energy source at this luminosity is gravitational contraction\\

Most of the gas remaining from the process of star formation is swept into interstellar space by a \bf{protostellar wind}.\\

Planets may form within the protostellar disk that surrounds a forming star.\\

Main-sequence Phase
\begin{itemize}
\item lasts about 10 billion years
\item surface radiates energy at the same rate that core generates energy
\item energy generated by nuclear fusion
\end{itemize}

Protostar Phase
\begin{itemize}
\item pressure and gravity not precisely balanced
\item energy generated by gravitational contraction
\item luminosity much greater than the sun
\item radius much larger than the sun
\end{itemize}

\bf{interstellar medium}: the gas and dust that lies in between the stars in the Milky Way galaxy\\
Interstellar clouds called molecular clouds are the cool clouds in which stars form\\
Most abundant in an interstellar molecular cloud: $H_2$ \\
\bf{Interstellar dust} consists mostly of microscopic particles of carbon and silicon \\
Part of electromagnetic spectrum generally giving best views of stars forming in dusty clouds: \bf{infared} \\
Looking by eye at a star near the edge of a dusty interstellar cloud. The star will look \bf{dimmer and redder} than it would if it were outside the cloud.\\
Most interstellar clouds remain stable in size because the force of gravity is opposed by \bf{thermal pressure} within the cloud.\\
A cold, dense gas cloud is most likely to give birth to star because this type of cloud has lower thermal pressure (due to the low temperature) and stronger gravity (due to the high density). \\
Core temperature required before hydrogen fusion can begin in a star: 10 million K \\
Smaller stars spend more time in the protostellar phase of life \\
Vast majority of stars in a newly formed star cluster are \bf{less massive than the Sun}\\
Brown Dwarfs:
\begin{itemize}
\item form like ordinary stars but are too small to sustain nuclear fusion in their cores
\item have masses less than about 8\% that of our Sun
\item supported against gravity by degeneracy pressure, which does not depend on the object's temperature
\end{itemize}
\bf{Radiation pressure} prevents stars of extremely large mass from forming\\

Stages of a \bf{high mass star} (first to last):\\
contracting cloud of gas and dust, protostar, main-sequence O Star, red supergiant, supernova, neutron star\\
Elements from first to last produced: \\
Helium, Carbon, Oxygen, Iron \\
The \bf{CNO Cycle} is the process by which hydrogen fusion proceeds in high-mass stars. \\
If you returned to our solar system in 10 billion years you would most likely see a white dwarf\\
High Mass Stars ($>8M_Sun$):
\begin{itemize}
\item have higher fusion rate during main sequence life
\item late in life fuse carbon into heavier elements
\item end in a supernova
\end{itemize}
Low Mass Stars ($<2M_Sun$):
\begin{itemize}
\item final form is a white dwarf
\item have longer lifetimes
\item end life as a planetary nebula
\end{itemize}
The core of a high-mass star shrinks and heats up after it runs out of hydrogen.\\
