\subsection{Assignment 1}
{\bf Distance from Earth} (furthest to closest): Andromeda, far side of the milky way, near side of the milky way, orion nebula, Alpha Centauri, Pluto, The Sun

If an object is 10 light-years away, then we see it as it was 10 years ago, but if it is 20 light-years away, we see it as it was 20 years ago. In other words, more distant objects have aged more since their light left on its way to Earth.

Timeline: Universe begins to expand, Elements such as carbon and Oxygen first form, nuclear fusion begins in the Sun, Earliest life on Earth

Our cosmic address: Earth, solar system, Milky Way Galaxy, Local Group, Local Supercluster, universe

We cannot See a Universe that is 20 billion years away because it is not in our observable universe

\subsection{Assignment 2}
The {\bf Tilt of a planet is responsible for seasons}. Jupiter (3*) has almost no seasons compared to Uranus (97*). Earth and Mars have a similar tilt.

\begin{tabular}{|c|c|}
\hline
Time of Year & Earth-Sun Distance \\ \hline
March (Northern Spring Equinox) & 149.0 million km\\
June (Northern Summer Solstice) & 152.0 million km\\
September (Northern Fall Equinox & 150.2 million km\\
December (Northern Winter Solstice) & 147.2 million km\\ \hline
\end{tabular}

Earth is actually farthest from the Sun when it is summer in the Northern Hemisphere. We conclude that variations in the Earth-Sun distance from are not the major cause of our seasons.

The {\bf Moon’s orbit about Earth is tilted} (by about 5°) with respect to Earth’s orbit about the Sun. As a result, the actual number of solar eclipses that occur each year is approximately 2 (instead of one each month)

Apparent {\bf retrograde motion can be observed by noticing changes in Mars's position among the constellations}. Note that a complete period of apparent retrograde motion unfolds while Earth moves a significant fraction of its orbit, which means it takes several months. Apparent retrograde motion occurs as Earth ``laps'' Mars in their respective orbits around the Sun. The middle of a period of apparent retrograde motion occurs when Mars is closest to Earth in its orbit and in a full phase as viewed from Earth, which is why it is brightest in our sky at that time. It is also directly opposite the Sun in the sky at that time, which is why it crosses the meridian at midnight.

The {\bf Greeks explained retrograde motion by imagining that planets moved around small circles that in turn moved around larger circles around Earth}. Because this model made reasonably accurate predictions of planetary positions and fit with other philosophical ideas that they held, the Greeks had no compelling reason to reject it,

{\bf Random}: How frequently does the Galactic Center (in the constellation Sagittarius) and the Sun align, that is, appear in the same constellation? Once A year

Neutrinos rarely interact with anything on Earth (faulty premise of movie 2012)

\subsection{Assignment 3}
{\bf Kepler's second} law tells us that as an object moves around its orbit, it sweeps out equal areas in equal times

{\bf Venus} is full whenever it is on the opposite side of the Sun from Earth, although we cannot see the full Venus because it is close to the Sun in the sky. For Venus to be high in the sky at midnight, it would have to be on the opposite side of our sky from the Sun. But that never occurs, because Venus is closer than Earth to the Sun.

{\bf Falsifiable}: (could be proven false)

True Statements belong in Sun centered and ``Both'' models

{\bf Earth Centered Model Only}
\begin{itemize}
\item A planet beyond Saturn rises in east and sets in West
\end{itemize}

{\bf Sun Centered Only}
\begin{itemize}
\item Positions of nearby stars shift back and forth slightly each year
\item Mercury goes thorough a full cycle of phases
\end{itemize}

{\bf Both Models}
\begin{itemize}
\item Moon rises in east and sets in west
\item A distant galaxy rises in the east and sets in the west
\item stars circle daily around north and south celestial pole
\end{itemize}

{\bf Neither Model}
\begin{itemize}
\item We sometimes see a crescent Jupiter
\end{itemize}

An object must come between Earth and the Sun for us to see it in a crescent phase, which is why we see crescents only for Mercury, Venus, and the Moon.

{\bf Greek geocentric model}, the retrograde motion of a planet occurs when the planet actually goes {\bf backward in its orbit around Earth }

{\bf Random:} Copernicus's Sun-centered model did not make significantly better predictions of planetary positions in our sky. (not an advantage of it)

{\bf Tides:}
\begin{itemize}
\item Any particular location on Earth has two high tides and two low tides each day.
\item Tides also affect land, although not as much
\item One tide bulge faces the moon, the other is away from
\item The second tidal bulge arises because gravity weakens with distance, essentially stretching Earth along the Earth-Moon line.
\item High tides are highest at both full moon and new moon
\item Low tides are lowest at both full moon and new moon.
\item Moon is larger factor than the sun because gravitational attraction between Earth and the Moon varies more across Earth than does the gravitational attraction between Earth and the Sun
\end{itemize}

\subsection{Assignment 4}
\begin{itemize}
\item The Sun emits all colors of visible light, but cooler gasses on the Sun's surface absorb some of these colors.
\item The most intense color in an absorption spectrum can tell us the temperature of an object.
\item {\bf Wien's Law}: thermal radiation from a higher object peaks at a shorter wavelength.
\item Electrons lose energy exactly equal to the proportion of distance between ``rings'' they jump, regardless of which rings they traverse.
\item Visible light and radio waves reach the Earth's surface, infrared light reaches mountain tops, UV light reaches the upper atmosphere, and X-rays don't enter the atmosphere.
\item The light-carrying area of a telescope varies at a rate double its diameter.
\item The Hubble Telescope has a resolution of less than 0.1 arcseconds.
\end{itemize}

\subsection{Assignment 5}
Jovian planets \underline{\hspace{5em}} than terrestrial planets:
\begin{itemize}
\item Are more massive
\item Are lower in average density
\item Are bigger
\item Orbit the Sun farther
\item Have more moons
\item Have rings
\end{itemize}

How are Pluto and Eris different from other planets:
\begin{itemize}
\item Smaller
\item More elliptical orbits
\item Less massive
\item Similar composition to comets (ice and rock)
\end{itemize}

Characteristics of planets in our solar system:
\begin{itemize}
\item Large bodies have orderly motions
\item There are exceptions to most trends
\begin{itemize}
\item Venus spins backwards
\item Uranus rotates on its side (axial tilt $\approx 90^\circ$)
\item The moon is about $\frac{1}{4}$ the size of Earth
\end{itemize}
\item Planets orbit the Sun in the same plane
\item Planets closer to the Sun move  around their orbits at higher speed than planets farther from the Sun
\item All the planet (not counting Pluto) have nearly circular orbits
\end{itemize}

Small bodies:
\begin{itemize}
\item Rocky = asteroids
\begin{itemize}
\item Found in the asteroid belt
\end{itemize}
\item Icy = comets
\begin{itemize}
\item Found in the Kuiper belt (starts around Neptune, extends past Pluto) and Oort cloud (sphere around the solar system, far beyond Pluto)
\end{itemize}
\end{itemize}

Detecting extrasolar planets
\begin{itemize}
\item Planets are very dim compared to their star, it makes it very hard to image them visually.  The angular separation from Earth is also very small.
\item We can detect planets by watching how their star ``wobbles'' due to their gravity (Doppler technique).  This technique can only give us the \textbf{minimum} mass of the planet (unless it's on the same plane as the Earth)
\item Smaller orbital radius of planet results in higher max speed of the star and shorter period of rotation
\item Mass of the planets only affects the max speed of the star in its ``wobble''
\item If the planets orbital plane lie between the Earth and the star we can see eclipses --- a slight dip in the stars luminosity
\item As of 2008, the most extrasolar planets have been discovered by the Doppler technique
\item The Kepler mission mostly looked for eclipses
\item Transit technique has the best chance of finding Earth-like planets
\item The astrometric technique uses careful measurements of positions of celestial bodies to find planets
\end{itemize}

Properties of extrasolar planets (discovered so far):
\begin{itemize}
\item Some jovian planets have been found closer to their star than Mercury is to the Sun (hot Jupiters)
\item Many are on very eccentric orbits
\item Jovian planets migrate closer to their star from their original orbits
\item Most are larger than Jupiter
\end{itemize}

Four process that shape planetary surfaces:
\begin{itemize}
\item Impact cratering
\begin{itemize}
\item number of impacts per square area about the same for all planets
\item Mostly occurs in the first few hundred million years after formation
\item Primary factor for craters still being visible is if they've been erased, because they were definitely there at some point
\end{itemize}
\item Volcanism
\begin{itemize}
\item Outgassing explains how terrestrial planets got their atmospheres
\end{itemize}
\item Tectonics
\begin{itemize}
\item Happens beneath the \emph{lithosphere}, the rigid layer of rock at the surface of a planet
\item Compression causes mountain ranges
\item Extension (stretching) causes cracks and valleys
\item Earth is the only planet where the lithosphere has been broken into plates
\item Lots of tectonic activity means lots of volcanic activity
\item Can only occur if the interior of the planet is liquid (hot!)
\item Big planets take longer to cool than small ones, so they have tectonic activity for longer
\end{itemize}
\item Erosion
\begin{itemize}
\item Occurs due to surface liquids, ices and gases
\item Liquid water is the best, causes much more pronounced features
\item Canyons (formed by glaciers), dunes, rock formations
\item Planet must be warm enough to have liquids, and big enough to capture an atmosphere
\end{itemize}
\end{itemize}
