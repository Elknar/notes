\documentclass[12pt]{article}
\usepackage{amsmath,amssymb,bookmark,mathtools,parskip,custom}
\usepackage[margin=.8in]{geometry}
\allowdisplaybreaks
\hypersetup{colorlinks,
    citecolor=black,
    filecolor=black,
    linkcolor=black,
    urlcolor=black
}
\setcounter{secnumdepth}{5}

\begin{document}

\title{CS 341 --- Algorithms}
\author{Kevin James}
\date{\vspace{-2ex}Winter 2015}
\maketitle\HRule

\tableofcontents
\newpage

\section{Solving Recurrences}
eg. mergesort
\[ T(n) =
\begin{dcases*}
2T\bigl(\frac{n}{2}\bigl) + \Theta(n) & if $n > 1$ \\
\Theta(1) & if $n = 1$
\end{dcases*}\]
which we can solve to find \[ T(n) \in \Theta(n\log n) \]

eg. stoogesort(A[1..n])
\begin{verbatim}
if n <= 1:
  return
if A[1] > A[2]:
  swap(A[1], A[2])
stoogesort(A[1..\frac{2n}{3}])
stoogesort(A[\frac{n}{3}+1..n])
stoogesort(A[1..\frac{2n}{3}])
\end{verbatim}

Let $T(n)$ be the runtime of stoogesort on $n$ elements. Then \[ T(n) =
\begin{dcases*}
3T\bigl(\frac{2n}{3}\bigl) + \Theta(1) & if $n > 1$ \\
\Theta(1) & if $n = 1$
\end{dcases*}\]

There are three methods for solving this recurrence:
\begin{enumerate}
\item Recursion Tree method: Expand for $k$ iterations to get a tree of terms, set $k$ to reach the base case, sum across rows, then over all levels.
\item Master method: Look up the answer. The Master Theorem: Let $T(n) = aT(\frac{n}{b}) + f(n)$ if $n > n+0$ and $c$ if $n = n_0$. Set $d = \log_b a$i and pick a small constant $\varepsilon > 0$. Case 1: $f(n) = O(n^{d-\varepsilon}) \implies T(n) = \Theta(n^d)$. Case 2: $f(n) = \Theta(n^d) \implies T(n) = \Theta(n^d \log n)$. Case 3: $\lim_{n\to\infty} f(n) / n^{d+\varepsilon} = \infty \implies T(n) = \Theta(f(n))$.
\item Substitution method: Guess the form of the solution $T(n) \leq x)$, then verify your guess by induction and fill in any constants.
\end{enumerate}

Example:
\begin{align*}
T(n) &= 2T\bigg(\frac{n}{2}\bigg) + n^2, 7 \\
T(n) &\leq cn^2 \\
n = 1: T(n) &= 7 \leq cn^2, c \geq 7 \\
T\bigg(\frac{n}{2}\bigg) &\leq c\bigg(\frac{n}{2}\bigg)^2 \\
T(n) &= 2T\bigg(\frac{n}{2}\bigg) + n^2 \\
     &\leq 2c\bigg(\frac{n}{2}\bigg) + n^2 \\
     &= \frac{2cn^2}{4} + n^2 \\
     &= \bigg(\frac{c}{2} + 1\bigg)n^2 \\
     &\leq cn^2, \frac{c}{2} + 1 \leq c, c \geq 2
\end{align*}
We can then pick $c = 7$ and solve as \[ T(n) \leq 7n^2 \implies T(n) \in O(n^2) \] We can also note that $T(n) \geq n^2$, so $T(n) \in \Theta(n^2)$.

Example:
\begin{align*}
T(n) &= 3T\bigg(\bigg\lfloor \frac{n}{2} \bigg\rfloor\bigg) + 4T\bigg(\bigg\lfloor \frac{n}{4} \bigg\rfloor\bigg) + 1, 1 \\
T(n) &\leq cn^2 \\
n = 1: T(n) &= 1 \leq cn^2, c \geq 1 \\
T\bigg(\bigg\lfloor \frac{n}{2} \bigg\rfloor\bigg) &\leq c\bigg\lfloor \frac{n}{2} \bigg\rfloor^2 \\
T(n) &= 3T\bigg(\bigg\lfloor \frac{n}{2} \bigg\rfloor\bigg) + 4T\bigg(\bigg\lfloor \frac{n}{4} \bigg\rfloor\bigg) + 1 \\
     &\leq 3c \bigg\lfloor \frac{n}{2} \bigg\rfloor^2 + 4c\bigg\lfloor \frac{n}{4} \bigg\rfloor + 1 \\
     &\leq 3c \bigg(\frac{n}{2}\bigg)^2 + 4c \bigg(\frac{n}{2}\bigg)^2 + 1 \\
     &= \bigg(\frac{3}{4} + \frac{4}{16}\bigg)cn^2 + 1 \\
     &= cn^2 + 1
\end{align*}
but since we could not get rid of the constant, we try
\begin{align*}
T(n) &\leq cn^2 - c_0 \\
n = 1: T(n) &= 1 \leq cn^2 - c_0, c \geq 1 + c_0 \\
T\bigg(\bigg\lfloor \frac{n}{2} \bigg\rfloor\bigg) &\leq c\bigg\lfloor \frac{n}{2} \bigg\rfloor^2 - c_0 \\
T(n) &= 3T\bigg(\bigg\lfloor \frac{n}{2} \bigg\rfloor\bigg) + 4T\bigg(\bigg\lfloor \frac{n}{4} \bigg\rfloor\bigg) + 1 \\
     &\leq 3\bigg(c \bigg\lfloor \frac{n}{2} \bigg\rfloor^2 - c_0\bigg) + 4\bigg(c\bigg\lfloor \frac{n}{4} \bigg\rfloor - c_0\bigg) + 1 \\
     &\leq 3c \bigg(\frac{n}{2}\bigg)^2 - 3c_0 + 4c \bigg(\frac{n}{2}\bigg)^2 - 4c_0 + 1 \\
     &= \bigg(\frac{3}{4} + \frac{4}{16}\bigg)cn^2 - 7c_0 + 1 \\
     &= cn^2 - c_0, -7c_0 + 1 \leq -c_0, c_0 \geq \frac{1}{6}
\end{align*}
Pick $c_0 = \frac{1}{6}, c = \frac{7}{6}$ and we see that \[ T(n) \leq \frac{7}{6}n^2 - \frac{1}{6} \implies T(n) \in O(n^2) \]

\section{Algorithm Design Techniques}
\subsection{Divide and Conquer}
Divide your problem into subproblems of the same type, then use recursion to solve each problem and combine the results.

{\bf Problem (Maxima)}: Given a set $P$ of $n$ points in 2D, we say point $p$ dominates point $q$ if and only if $p$ has both a greater $x$ and $y$ value than $q$. We say point $q$ is maximal if and only if $q \in P$ and no point in $P$ dominates $q$. Find all maximal points.

{\bf Solutions}:
\begin{itemize}
\item Brute Force: for each $q \in P$, check if no points dominat $q$. Total time: $\Theta(n^2)$
\item Divide and Conquer: divide into two subarrays of size $\frac{n}{2}$.
\item Divide by Medians: instead of dividing by size, divide by a median vertical line.
\end{itemize}

\begin{verbatim}
maxima(sorted[p_1..p_n]):
  1. if n == 1: return p_1
  2. [q_1..q_l] = maxima([p_1..p_{n/2}])
  3. [s_1..s_m] = maxima([p_{n/2}..p_n])
  4. i = 1
  5. while q_i.y > s_1.y
  6.   i += 1
  7. return [q_1..q_l, s_1..s_m]
\end{verbatim}
which is an $O(n\log n)$ algorithm.

{\bf Problem (Closest Pair)}: Given set $P$ of $n$ points in 2D, find a pair $p, q \in P$ such that these points have a smaller distance between them than any other pair of points in $P$, ie. $d(p,q) = \sqrt{(p.x - q.x)^2 + (p.y - q.y)^2}$.

{\bf Solutions}:
\begin{itemize}
\item Brute Force Algorithm: $\Theta(n)$
\item Shamos' Algortithm: $\Theta(n\log^2 n)$. Note that if we refine the Shamos algorithm by pre-sorting $P$ also by y-coordinate at the beginning, we find that our complexity becomes $O(n\log n)$.
\end{itemize}

\begin{verbatim}
def bruteForce(P):
  distance = infinity
  for each p in P:
    for each q in P (q neq P):
      distance = min(distance, d(p,q))
  return distance

def shamos(P):
  if n <= 10:
    return bruteForce(P)
  x_m = median x_coordinate
  P_L = { p in P : p.x < x_m }
  P_R = { p in P : p.x > x_m }
  d_L = shamos(P_L)
  d_R = shamos(P_R)
  d = min(d_L, d_R)
  <p_1..p_m> = sorted_by(points in { p in P : x_m - d <= p.x <= x_m + d }, y)
  for i = 1 to m do
    j = i + 1
    while p_j.y <= p_i.y + d:
      d = min(d, d(p_i, p_j))
      j++
  return d
\end{verbatim}

\subsubsection{Multiplying Large Numbers}
Given two $n$-bit numbers $A = a_{n-1}, a_{n-2}, \dots a_0$, $b = b_{n-1}, b_{n-2}, \dots b_0$ in binary, compute $AB = c_{n-1}, c_{n-2}, \dots c_0$.

The ``Elementary School'' algorithm for this involves doing
\begin{verbatim}
    1011
  x 1101
--------
    1011
  1011
 1011
--------
10001111
\end{verbatim}
which is $O(n)$ shifts and $O(n)$ additions, thus making it an $O(n^2)$ algorithm.

The ``Karatsuba and Ofman'' algorithm involves a different process:
\begin{verbatim}
A' = [a_{n-1}..a_{n/2}]
A" = [a_{n/2 - 1}..a_0]
A = [A'..A"]

B' = [b_{n-1}..b_{n/2}]
B" = [b_{n/2 - 1}..b_0]
B = [B'..B"]

AB = A'B'2^n + (A'B" + A"B')2^(n/2) + A"B"
\end{verbatim}
which gives us a complexity of $O(n^2)$.

\end{document}
